\subsection{School Leadership Criterion}
The school leadership (1) makes decisions to facilitate actions that focus the energies of the school on student achievement of the schoolwide learner outcomes, i.e., global competencies, (2) empowers the staff, and (3) encourages commitment, participation, and shared accountability for student learning in a global environment.

\subsubsection{Defined Responsibilities, Practices, etc.}

\indicator{The school has administrator and faculty written policies, charts, and handbooks that define responsibilities, operational practices, decision-making processes, and relationships of leadership and staff.}

\prompt{Evaluate these administrator and faculty written policies, charts, and handbooks. Determine the clarity and understanding of these by administration and faculty.}

\begin{findings}
CMIS has clear policies that define responsibilities, operational practices, decision making processes and relationships of leadership and staff. 

These policies and procedures can be found in the CMIS employment contracts and the \href{https://drive.google.com/a/cmis.ac.th/file/d/0ByVFfrm0zfolVm9uc19UNl82NVpPMUZ1ZEstZlFidEY1c2hn/view?usp=sharing}{CMIS Faculty Handbook}. They are reviewed annually during the administrative, new teacher, and returning staff orientations. 

Furthermore, a list of administrative teams is included in the Faculty Handbook and CMIS Student Planner. 

CMIS organized ``\href{https://drive.google.com/a/cmis.ac.th/file/d/0ByVFfrm0zfolbHNvSWhVWmJYU3M/view?usp=sharing}{Teacher Table Talk}'' sessions during early release days to capture insights and suggestions about the decision-making processes and relationships of leadership and staff. 

In November, 2016 the \href{https://docs.google.com/a/cmis.ac.th/document/d/1iW_tWIwRlWU2p0oIOvd3usDsxj9qYDt_2ROwNPBTHSc/edit?usp=sharing}{Teacher Leadership Team}  which is made up of teachers and administrators, participated in the revision of the handbook format and made it available in an electronic formatted version. We plan to make this an annual project as it was previously revised only by administrators.

CMIS Board governance requires the Executive Team (i.e. Superintendent, Director and Manager) to maintain up-to-date operational handbooks and procedural guidelines.

Board member roles and responsibilities are clearly outlined and described in the \textit{Christian Churches of Thailand Board Handbook}.

\minor{So what...}

CMIS Leadership has clearly defined roles for teachers only through a variety of methods, most notably the Teacher Handbook. CMIS should continue to maintain, monitor, and modify, as necessary, the Teacher Handbook with stakeholder feedback and input. Additionally, CMIS should develop a Leadership Handbook to clearly outline the roles and responsibilities of principals, directors, and team leaders.
\end{findings}

\subsubsection{Existing Structures}

\indicator{The school has existing structures for internal communication, planning, and conflict resolution.}

\prompt{How effective are the existing structures for internal communication, planning, and conflict resolution?}

\begin{findings}

CMIS has many existing structures for internal communication, planning, and resolving differences.

Communication and Planning Examples:
The School Executive Team (SET) shares information from the monthly Board of Director Meetings to the School Management Team (SMT), and to the Teacher Team Leaders who communicate the information to the teachers.

Internal communication is also highlighted through Weekly HS Principal Notes, MS Principal Message and ES Principal Notes that are sent electronically to all staff.

The Superintendent and principals are out on campus every morning before school for personal communication with students, parents and staff.

The Superintendent and principals meet weekly to address any community concerns and plan.

Additionally, e-mails, memos, google docs are used to further enhance communication across the board.

Communication Examples:
Internal communication and planning is included at monthly staff meetings are held every Wednesday for K-12 Teams, ES, MS and HS Divisions, Teacher Leadership Teams, and Early Release Professional Development Trainings.

PTG Team Leaders meet with the Superintendent monthly to discuss parent concerns or suggestions for school improvement.

StuCo (Student Council) and the Superintendent meet for lunch each semester to discuss student suggestions and ideas for school improvement.

Conflict Resolution Examples:
CMIS organizes Teacher Table Talk sessions during early release days to capture insights and suggestions about the existing structures for resolving differences.

There is an anonymous process for teachers to share concerns with administration through the Teacher Administration Communication Team (TACT) that meets monthly to review and address current staff concerns. The concerns are addressed collaboratively and a summary of the solutions and suggestions are e-mailed out to all staff.

If teachers have concerns related to their students they are encouraged to reach out to the Student Success Team facilitated by our Student Service Coordinator. There is a Student Service Request Form available on the teacher dashboard section of the CMIS Handbook.This form activates immediate support from the divisional counselor to meet with the teacher and create a plan of action.

There is also a formal structure in place with regard to resolving internal conflicts between teachers and administration  through the board of directors (Staff Grievance Policy). This policy is included in the CMIS Faculty Handbook.

Parents with student or school concerns are directed to page 26 of the Student Planner: Communication between School and Home. 

Parents are also encouraged to share their concerns or suggestions with the superintendent during the weekly Super Thursdays or a PTG Leadership Team member at any time. 

Request For Support Forms are used by staff via-the Student Success Team when in need for conflict resolution with students.

\minor{So what...}
CMIS clearly maintains structures for planning, communication, and conflict resolution. CMIS Leadership should maintain and monitor these process and evaluate their effectiveness. 

\end{findings}
