
\tcbsection{CMIS RESPONSE TO RECOMMENDATIONS FROM PREVIOUS REPORT }
\subsection{Item 1}

\minor{Action Plan}

Continued development of the written, taught and assessed curriculum at all levels, with the goal of a comprehensive curriculum document that includes standards and benchmarks, assessment tools, resources, technology integration and sample units for all subject areas and grade levels. This would be evaluated at all levels to check for coverage and continuity, as well as to ensure that Educational Objectives were being thoroughly covered in our program. It would also include a regular evaluation and improvement system being put in place and maintained.

\begin{longtabu} to \textwidth {|X|X|}
\hline
\textbf{Growth Area 2011} & \textbf{Steps Taken Since Interim Report 2014} \\
\hline
1. \minor{Curriculum} EOs (ESLRs): Reflect on the appropriateness of the Educational
Objectives, develop processes for alignment and integration into the curriculum, instruction and school activities, and create appropriate measurements for student mastery. &

\parbox[t]{2.8in}{
Modify EOs (ESLRs) to SLOs\\
Staff involvement in SLOs creation \\
Modification of EOs into measurable SLOs\\
Integration of SLOs with UbD units\\
Instructional Round data collection of SLOs applications \\
Implementation of SLOs into student communication group activities\\
Student created visuals of SLOs }\\
\hline

2. \minor{Curriculum} curriculum development and documentation - Continued development of a standards based, comprehensive, progressive, written curriculum in elementary social studies, science, and non-core courses in the arts, as well as continued development of secondary courses to ensure a CMIS curriculum (including curriculum maps) are in place and is sustainable; and develop a schedule for regular assessment and revision of curriculum through a curriculum renewal cycle and process. &

\parbox[t]{2.8in}{
Unpacking of CCSS ELA, CCSS Math, NGSS, and C3\\
Training on Understanding by Design (whole group, small group)\\
Implementation of UbD Unit Menu\\
Modification and use of UbD template to reflect current professional practice\\
Development and use of Adoption/Resource Cycle \\
Development and use of research based adoption vetting instruments\\
Development and use of Scope and Sequence Blueprints K-12\\
Collection of best practices in Summative Assessment development \\
Development for and implementation of Datawise process \\
Use of Instructional Rounds observational process\\
Use of Teach for Success observation protocol\\
Implementation and training of staff for Datawise data analysis process\\
Training of staff for Middle School ELA/SS Alignments and PARCC Blueprints\\
Training of staff for High School ELA PARCC Blueprints\\
Development and implementation of UbD Interrater Reliability Practicum} \\
\hline

3. \minor{Curriculum} instruction and assessment - Utilize current and research-based instructional strategies systematically and with the support from the leadership. Implement a schedule for reviewing  assessment practices (formative and summative) and results to inform curricular and instructional decisions; and implement grade level common assessments (standardized or otherwise) for levels that currently do not have them. &

\parbox[t]{2.8in}{
Development and implementation of Formative Assessment training\\
Development and implementation of Summative Assessment Peer Reflection (for HS only)\\
Modification and use of UbD template to reflect current professional practice in formative and summative assessment\\
Modification and use of UbD template to reflect current professional practice in literacy and rigor (e.g. close reading, DOK)\\
Development and implementation of school wide writing pre assessment \\
Development and implementation of common ELA writing rubric for argumentative writing\\
Development and implementation of Looking at Student Work protocols (e.g. Critical Friends, Longfellow Slice)\\
Implementation of MAP Assessment for grades 2-9 \\
}
\\
\hline
\end{longtabu}

\subsection{Item 2}

Develop a process of continuously evaluating and prioritizing foreign teacher
compensation and benefits to encourage them to extend their stay in CMIS. This may also include evaluating teacher workloads, schedules, recognition opportunities and extra responsibilities including substituting. This will include forms of evaluation and follow-up and a system for staff and teacher evaluation that is comprehensive and transparent. In addition, effective procedures for due process, renewal of contracts, and grievances will be developed.

\begin{longtabu} to \textwidth {|X|X|}
\hline
\textbf{Growth Area 2011} & \textbf{Steps Taken Since Interim Report 2014} \\
\hline
4. \minor{Professional Development} Further implement the professional development plan so professional development is focused on prioritized and definable school curricular needs as seen in Growth Areas 1-3. Professional development needs to purposeful and sustained to improve student learning.
 &

\parbox[t]{2.8in}{
Unpacking of CCSS ELA, CCSS Math, NGSS, and C3\\
Development and delivery of training on Key Instructional Shifts in ELA\\
Development and delivery of training on Key Design Elements in CCSS\\
Development and delivery of training on Formative Assessment\\
Development of Summative Assessment Peer Reflection \\
Development and delivery of  training on Text Complexity\\
Training on UbD Essentials (outside consultant)\\
Development and delivery of training on Building Knowledge Systematically in CCSS\\
Development and delivery of training on Close Reading \\
Development and delivery of training on Student Engagement Essentials \\
Development and delivery of training on Text Dependent Questions\\
Development and delivery of training on Tier 2 and 3 Vocabulary development\\
Development and delivery of New Teacher Orientation modules\\
Development and delivery of training on SMART goals\\
Development and delivery of training on NGSS through use of vetting instruments\\
Development and delivery of training on CCSS  Mathematics through use of vetting instruments\\
Development and delivery of training on C3 unpacking, Scope/Sequence, and Unit Creation using IDM  }\\
\hline

Teacher compensation and evaluation: Further refine and implement a process of continuously evaluating and prioritizing foreign teacher compensation and benefits to encourage them to extend their stay at CMIS. This may also include evaluating teacher workloads, schedules, recognition opportunities and extra responsibilities, including substituting. This will include forms of evaluation and follow-up and a system for staff and teacher evaluation that is comprehensive and transparent. In addition, effective procedures for due process and renewal of contracts and grievances must be developed.
 &

\parbox[t]{2.8in}{
Increase staff salary step and housing benefit\\
Creation of staff attendance incentive\\
Creation of Teacher Leadership Team\\
Increase in Teacher Leadership stipend\\
Creation of stipends for staff coaches \\
Creation of professional development fund for all staff\\
Creation of new Staff Welcome Committee\\
Creation of New Hire Buddy Program\\
Creation of New staff handbook \\
Creation of New staff orientation \\
Creation of Free Certified First Aid Training for volunteer foreign staff\\
Workload analysis data and workload payment committee\\
Scheduling modification to increase staff collaboration time\\
Creation of Staff Grievance Policy (formal and informal)\\
Annual TACT meeting for salary and benefit every 2 years\\
Creation of Monthly Kudos Cup (staff appreciation)\\
Creation of specific timeline for renewal of contracts ( end Jan.)\\
Creation of simplified contract template for 2017-18\\
Creation of exit interview for leaving staff\\
Creation of Substitute coordinator position on campus\\
Extra pay for staff using preparation time for substitution\\
Use of Instructional Rounds observational process\\
Use of Teach for Success observation protocol\\
Implementation and use of The Essential Practices of High Quality Teaching and Learning\\
Development and delivery of training on Depth of Knowledge } \\
\hline
\end{longtabu}

\subsection{Item 3}

Develop administrative and Board policies and procedures that are well documented, clearly communicated and effective for the daily running of the school in regards to the school board, principals, business manager, and director.

\begin{longtabu} to \textwidth {|X|X|}
\hline
\textbf{Growth Area 2011} & \textbf{Steps Taken Since Interim Report 2014} \\
\hline
5. \minor{Board} policies and procedures - Develop School and Board policies and procedures that are well documented, clearly communicated and effective for the daily running of the school in regard to the Board, principals, business manager, and director. Furthermore, policies and decision-making processes must be visible and transparent to all stakeholders.
 &

\parbox[t]{2.8in}{
Creating a specific CMIS Board Handbook, \\
Creation of Board policies: Principles of Good Practice and Individual Board Member Principles of Good Practice \\
Modification of Faculty Handbook format\\
Updated policies and procedures reviewed annually at staff orientations \\
Implementation of Wednesday staff meetings\\
Creation of Board meeting summaries available to community\\
Implementation of Parent Coffee Mornings at start of the school year\\
CMIS Board/SET members attend monthly PTG Meetings\\
Board/SET members are involved with the New Teacher Orientation day\\
Board/SET members participate in FOL/WASC focus groups \\
Board generates annual community letters  }\\
\hline

6. Board training: To help ensure sound governance, the board should develop a process for consistent board training and orientation as well as board self-evaluation.
 &

\parbox[t]{2.8in}{
Creation of Board retreats for training, orientation and self-evaluation. \\
Utilization of outside consultant John Ritter worked with Board \\
CMIS membership National Association of Independent Schools \\
Creating new Board member mentorship and training } \\
\hline
\end{longtabu}

\subsection{Item 4}

Create a school wide master plan which will include planning for the development of new or existing property, a budget and resource acquisition system that aids educational planning for departments, IT equipment, books, materials, other resources, and capital improvements.

\begin{longtabu} to \textwidth {|X|X|}
\hline
\textbf{Growth Area 2011} & \textbf{Steps Taken Since Interim Report 2014} \\
\hline
8. \minor{Resources, facility, master plan} Further develop a school-wide master plan that includes planning for the development of new or existing property, a budget and resource acquisition system that aids educational planning for departments, IT equipment, books, materials, other resources, and capital improvements. The plan should include involvement of all stakeholders to help ensure the planning process is transparent.
 &

\parbox[t]{2.8in}{
Creation of Campus Development Plan\\
Analysis of community survey plan\\
Development and implementation of 10 year Adoption/Resource Cycle. \\
Development and implementation of Resource Request and Resource Renewal procedures and forms \\
Development and implementation of vetting and evaluation instruments for resource adoption   }\\
\hline

\end{longtabu}

\subsection{Item 5}

 Continue to improve and expand communication among all stakeholders at CMIS.

\begin{longtabu} to \textwidth {|X|X|}
\hline
\textbf{Growth Area 2011} & \textbf{Steps Taken Since Interim Report 2014} \\
\hline
9. \minor{Communication} CMIS needs an effective communication system to inform all stakeholders of school processes and procedures, decision-making responsibilities, and school-wide issues, and information including student progress.
 &

\parbox[t]{2.8in}{
Increase in Facebook/social media community communication\\
Creation of Administration Parent newsletter\\
Implementation of Emergency messages through SMS\\
Implementation of Power School communication alerts\\
Greater use of Google Classroom and guardian access to Google Classroom \\
New formated website with greater emphasis on finding information and weekly communication. 
Weekly faculty notes  }\\
\hline

\end{longtabu}

\subsection{Item 6}

Expand our community connections to include an alumni association and development office to increase opportunities to raise support from places other than just tuition.

\begin{longtabu} to \textwidth {|X|X|}
\hline
\textbf{Growth Area 2011} & \textbf{Steps Taken Since Interim Report 2014} \\
\hline
NA
 &

\parbox[t]{2.8in}{
Creating a fund development committee to increase communication,\\
facilitate volunteer involvement, and encourage philanthropic commitment to CMIS.\\
Utilizing an alumni coordinator to develop and maintain an effective alumni database. }\\
\hline

\end{longtabu}

