\documentclass{report}
\title{WASC Self-Study Report}
\date{November 16, 2016}
\author{CMIS}
\usepackage{tabularx,colortbl,array}
\usepackage[labelfont=bf]{caption}
\usepackage[table]{xcolor}
\usepackage{graphicx}
\usepackage{pgfplots}
%\usepackage{paracol}
\usepackage{blindtext}
%\usepackage{capt-of}
\usepackage{subfig}
\usepackage{enumitem}
\usepackage{verbatim}

\usepackage{float}
\usepackage{wrapfig}
\floatstyle{boxed} 
\restylefloat{figure}
\usepackage{fancyhdr}
\pagestyle{fancy}
\fancyhf{}
\rfoot{Page \thepage}
\cfoot{\leftmark}
\usepackage{xcolor}
\usepackage{hyperref}
\hypersetup{%
  pdftitle=WASC Self Report,
  colorlinks=true,
  urlcolor=cyan,
  urlbordercolor=magenta
}
\setlength{\parskip}{1em}

\makeatletter
\ifnum\inputlineno=\m@ne
\let\showlineno\@empty
\else
\def\showlineno{ line \the\inputlineno}
\fi
\makeatother

\usepackage[most]{tcolorbox}
\usepackage{marginnote}
\usepackage{booktabs}
\newcommand{\debug}{%
        \marginnote{%
                \begin{debugBox}%
                        \jobname\showlineno%
                \end{debugBox}%
        }%
}

\newtcolorbox{evidenceBox}{%
  colback=black!5,
  colbacktitle=violet,
  boxrule=1pt,
  top=1em,
  breakable,
 % arc=4pt,
 % outer arc=4pt,
  title=Evidence
}

\newtcolorbox{findingsBox}{%
  colback=white,%black!5,
  colbacktitle=red,
  boxrule=1pt,
 % boxsep=1pt,
 % top=1em,%-0.5\baselineskip,
  breakable,
 % arc=4pt,
 % outer arc=4pt,
  title=Findings
}

\newtcolorbox{debugBox}{%
        colback=yellow,
        title=DEBUG
}
        

\newenvironment{findings}
{
\debug
\noindent\textbf{Findings}

}
{
%\end{findingsBox}
}

\newenvironment{evidence}
{
\debug
\begin{evidenceBox}
\vspace{-\topsep}
\begin{itemize}[leftmargin=*]
  \setlength{\parskip}{0pt}
  \setlength{\itemsep}{1pt}
}
{
\end{itemize}
\end{evidenceBox}
%\vspace{4em}
}


\newcommand{\indicator}[1]{ 
\noindent\textbf{Indicator:} #1
}

\newcommand{\prompt}[1]{ 
\noindent\textbf{Prompt:} \textit{#1}
}

\newenvironment{tbl}[2]
{
%\rowcolors{2}{lightgray}{white}
\tabularx{#1}{#2}
%\rowcolor{darkgray}
}
{
\endtabularx
}

\begin{document}
\maketitle

\tableofcontents
\chapter{Preface}
\chapter[Student/Community Profile Data]{Student/Community Profile and Supporting Data and Findings}
\section{CMIS Identifying Information}
\begin{description}
\item[Superindtendant]
Nel Capadona
\item[Year Established]
1954
\item[Last WASC Accreditation]June 2011
\item[Grades Accredited]PK-12
\item[Current Enrollment (May 1, 2016)] 498
\item[Enrollment At Last Accreditation (2010-2011)] 459
\item[Enrollment At Previous Accreditation (2004-2005)]401
\item[Current Number Of Teaching Staff (May 1, 2016)]69
\item[Number Of Teaching Staff At Last Accreditation (2010-2011)]57
\item[Number Of Teaching Staff At Previous Accreditation (2004-2005)]43
\end{description}


\section{CMIS: The School and the Community}
Chiang Mai International School (CMIS) is located just outside the old city of Chiang Mai, within the boundaries of the Superhighway. Chiang Mai is the largest significant city in Northern Thailand, and the former capital of the Lanna kingdom.  

Founded in 1296 AD, Chiang Mai is a growing city with a unique balance of modern conveniences and historic culture.  It is located in the northern region of Thailand, approximately 720 km from Bangkok.  Chiang Mai covers an area of 20,170 sq. km. and supports a population of 1,682,164 (based on June 2015 data).  It is a predominantly agricultural area with a well-established manufacturing and service-based economy.  Because of the rich culture, pleasant climate, stable economy, and friendly, relaxed atmosphere, Chiang Mai attracts many expats.  Tourism is one of the major industries, attracting more than 2 million foreign visitors annually.  The area also supports local and international businesses, NGOs, and mission organizations, which employ foreign professionals.

Our school, the first international school in Thailand outside of Bangkok, has a long and provincial history.  Our campus is a testament of this history!  From the beginning as an American Presbyterian missionary residence, to now as a 21st century school campus, the land on which CMIS sits has continually developed is a living and learning community with a strong sense of identity and a vision of educational excellence.  

Missionaries returning to work with the Church of Christ in Thailand (CCT) after World War II established a school for their children in Chiang Mai.  When CMIS was founded in 1954, the school was originally established in the McGilvary house (located on the First Church property along the Ping River).  Classes began with eight students on June 1, 1954.  In 1958, construction began on the present campus for the “Chiang Mai Children’s Center”.   Student numbers grew as more expatriate families seeking an English-language education for their children moved to Chiang Mai.

In 1984, representatives of the Thai Foreign Ministry and the CCT agreed that the formal establishment of an international school in Chiang Mai was a necessary step to achieving the school’s legal status.  Classes began in September 1985 for Kindergarten to Grade 8 under the new name “Chiang Mai International School” (CMIS).  High school grades were progressively added from 1992 to 1995.  

Our current campus is located in close proximity to private Thai schools, a hospital, a seminary and a university, all of which were founded by American Presbyterian missionaries and owned by the CCT.  The existing buildings on the CMIS campus, and their history of construction, are as follows: 
\begin{enumerate}
\item McKean House (Administration Building) (1906)
\item Pre-School Building (1958)
\item Library Building (1958)
\item Auditorium Building (1988)
\item High School Building (1990)
\item Elementary Building (1997)
\item Gymnasium Building (2007)
\end{enumerate}
Today CMIS is a dynamic international school with over 500 students, but is it is still small enough to retain a friendly and relaxed campus environment.  It also remains true to the traditions of its founders, serving missionary families and maintaining the heritage and values of the Christian faith at the heart of the school, while welcoming children of all faiths, cultures, and ethnic backgrounds from the growing international community in Chiang Mai.  

Of the five main international schools in Chiang Mai, CMIS is the only one in close proximity to the center of the city.  With the growth potential promised by the ASEAN agreement, CMIS has looked toward expanding the current campus to meet the needs of the community while maintaining its characteristic close family atmosphere.  Short and long term campus development planning have been a huge focus for the past six years. There is now a long term plan that includes additional buildings and planned renovations to existing buildings.  The timeline for development began in July 2014, and actual construction began in April 2016.  Construction of a new high school building, a covered court, a new library and cafeteria, and swimming pool are expected to be completed by December 2018.  Campus Development Timeline.  There are ongoing renovation and enhancement projects to maintain the school grounds and existing facilities.   Campus Development Renovation.   CMIS Campus Development.  CMIS Campus Development Site plan

As a standards-based American curricular school, CMIS offers a challenging educational experience, rooted in Christian values, which helps develop students into global citizens; as reflected in the CMIS vision, mission statement, and learner outcomes:

\minor{Student Characteristics}

Excellence in academics and ability to form successful relationships in a multicultural
Environment.


 \minor{School Identity}

 Developing academic excellence within a multicultural environment committed to Christian Values.
\section{CMIS Vision}

Educational excellence in a caring community, committed to Christian values, equipping international students for lives of learning and positive contributions, both locally and globally.

\section{CMIS Mission}

CMIS instills in students the capacity to effectively identify and pursue personal and academic goals based on educational excellence and strong moral foundations.
\begin{enumerate}
\item Academic opportunities are designed to ensure student readiness for college, career, and life.
Learner Outcomes:
\begin{itemize}  
\item At CMIS, students will learn to:
\item Embody a work ethic that values learning and academic integrity
\item Exhibit thinking that is creative and takes risks 
\item Pursue personal growth as adaptive, independent learners
\item Utilize resources and technology to effectively support learning and work
\end{itemize}
\item Positive community participation is emphasized in daily life at CMIS in order to ensure the continuing development of lasting, deeply-rooted student character.
Learner Outcomes:
At CMIS, students will learn to:
\begin{itemize}
\item Understand Christian virtues and character
\item Demonstrate integrity through consistent respect for people of all faiths
\item Build cultural awareness and an appreciation for diversity
\item Serve as responsible, proactive members of the global community
\end{itemize}
\end{enumerate}

\section{Enrollment Patterns}

Overall student enrollment has increased incrementally by 10 to 20 students per year over the past five academic years.  More than half of the increase came through filling existing vacancies in the Elementary School.  While the Elementary grew at an average rate of 6\% per year, total school enrollment increased by only 2\% per year.  Prior to 2012, Grade 6 was classified as Elementary, whereas from 2013 to present, Grade 6 is classified as Middle School.  Despite the categorical change, Middle School enrollment actually declined slightly throughout the reporting period.  In the 2016-2017 academic year, total school enrollment surpassed 500 for the first time in history.

\placeholder{Enrollment 2012-2016 graphs and data}

\section{Student Diversity}

The CMIS student body represents a diverse international community, with students from over 30 different countries.  The largest single group are those who have dual passport status or parents from two different countries.  Many factors are taken into consideration when selecting our CMIS students.  Age, gender, and nationality are certainly among them.  However, academic potential, the ability to follow instructions, personal maturity, social skills, and overall potential to succeed are of greater importance.  Regardless of nationality or ethnic identity, all successful applicants must be qualified personally, socially, and academically.  

We give priority enrollment to qualified applicants from the Western world.  There are no restrictions placed on our dual citizens or those coming from Western countries, such as the United States, Canada, the UK, Australia, New Zealand, and English-speaking countries in Europe or Asia.  We do not make exceptions for our missionary or diplomatic communities, but we do reserve spaces for them and give them priority for acceptance, if qualified.  In the 2016-2017 academic year, the percentage of American, Australian, Canadian, Dual, and United Kingdom students is as follows:

\begin{itemize}
\item American = 117 (23\%)
\item Australian / New Zealander 21 (4\%)
\item Canadian 8 (2\%)
\item Dual (\%)
\item United Kingdom 26 (5\%)
\end{itemize}

Thai, Korean, Chinese, and Japanese applicants are considered on a space-available basis.  Our target goals for each of these populations compared to our current enrollment are as follows:
\begin{itemize}
\item Thai 25\% = 183 (36\%)
\item Korean 20\% = 75 (15\%)
\item Chinese 10\% = 24 (5\%)
\end{itemize}
\placeholder{ MORE ENROLLMENT DATA GOES HERE }

\section{Criteria for Admission of Students}

\subsection{Standards for Entry}

CMIS strives to maintain a balanced, harmonious international environment where English is the language of inclusion.  We welcome applications from international students who:
\begin{enumerate}
\item Qualify academically (as determined by school records and standardized entrance assessment results) and
\item Meet the required English-language proficiency expectations for their grade level (as determined by CMIS guidelines).  
\end{enumerate}
CMIS strives to keep class sizes small, and thus provide the individualized, differentiated instruction that is needed to help each student succeed.  

Application to CMIS is competitive, and all students must have a minimum level of English language proficiency before they can be considered for enrollment.  The primary conditions for acceptance are academic eligibility, English-language proficiency, and exemplary behavior.  As part of the application process, applicants must submit copies of the current and previous year’s grade reports with teachers' comments, and any relevant standardized test results.  We can accept a limited number of academically qualified non-native English speakers, provided that their English-language proficiency falls within the CMIS guidelines for English as a Second Language (ESL) or English for Academic Purposes (EAP).  We can also accept a limited number for students with mild-to-moderate learning disabilities (as determined by previous school records, standardized testing, and Individualized Educational Plans), provided their disability falls within the CMIS guidelines for Learning Support (LS) and there is space available.

CMIS offers a Pre-School program for 4-year-old children who have reached their 4th birthday before the start of school in August.  The age / grade standard is set accordingly throughout elementary; age 5 for entry into Kindergarten, age 6 for grade 1, etc.  All applicants for Pre-School through Gr. 5 must have passed their required birth day by the start of school in August.  Students with birth dates after the start of school in August are classified according to their age at the time of enrollment, thus, any single CMIS grade level may have as much as 11 months of variability in the age of the students.  

For students who apply for entry to Middle School, applicants must have successfully completed the previous academic level and be within the appropriate age range for entry.  The age / grade level standard is adhered to as closely as possible, although an otherwise qualified applicant would rarely be asked to repeat a grade level that has been successfully completed in a comparable academic system.  Transferring into CMIS High School is dependent upon the number of credits on the student's transcript, compared to the CMIS graduation requirements.

Although we expect to have annual vacancies at each grade level, we reserve spaces for qualified, priority applicants.  Our Priority Categories are as follows: 
\begin{itemize} 
\item Christian Missionary families,
\item Diplomatic / Consular families,
\item NGO / Non-profit organization families,
\item Siblings of current CMIS students, and
\item Former CMIS families who are returning from abroad
\item CMIS Staff children
\item All other applicants are considered on a space-available basis.
\end{itemize}

CMIS uses the Early Screening Inventory (ESI) for assessing Pre-School and Kindergarten applicants.  The test is scaled for children 3 years and 6 months old to 6 years 11 months.  Applicants for entry into Grades 1 through 11 are assessed with the WIDA (World-class Instructional Design and Assessment) to determine English-language proficiency.  The WIDA is divided into four grade level clusters:

\begin{itemize}
\item Grades 1 - 2
\item Grades 3 - 5
\item Grades 6 - 8
\item Grades 9 - 12
\end{itemize}

Each form of the WIDA assesses the four language domains of Listening, Speaking, Reading, and Writing.  

\subsection{Admissions Committee Review Process}

The CMIS Admissions Committee is responsible for making all decisions regarding student applications. The Committee consists of the Admissions Director, School Superintendent, Dean of Students, Elementary Principal, Elementary Counselor, Middle School Counselor, and High School Counselor.    Learning Support Specialists serve as adjunct members of the Committee, and may be asked to comment, provide additional assessment, or recommend further testing.  

In their assessment of an application, the Committee takes the following information into account:

\begin{itemize}
\item applicant's profile
\item previous academic background
\item performance on CMIS entrance assessment
\item English-language ability
\item availability of space at the recommended grade level / support program”
\end{itemize}

In brief, applicants are expected to:

\begin{itemize}
\item qualify academically (above average grades in a comparable academic system)
\item meet the English-language proficiency expectations for their grade level
\item fit into a CMIS priority category or add value to our CMIS community
\item be able to succeed academically without ESL or LS, or fit our criteria to qualify for those support programs
\item meet behavioral and social expectations of the CMIS student body
\end{itemize}

 The Committee determines the applicants’ qualifications for each criteria by:

\begin{itemize}
\item reviewing academic records (minimum of 2 years, as appropriate) and standardized test results
\item conducting entrance assessments
\item reviewing psychologists' assessments or other supporting documentation
\item meetings with the applicant and family
\item reviewing letters of recommendation
\end{itemize}

CMIS uses Open Apply, an online application system.  Applicants and enquirers are encouraged to register their interest, schedule an appointment for a meeting with the Admissions Director, and to complete their application electronically.  

Newly admitted students’ families are given a New Family Survey to help identify areas we need to focus; both in terms of the student application process and advertising. We use this data to refine how we reach our target audience and provide a better experience for our new families.

\section{Student Post-Graduation Plans}

CMIS graduates gain admission to colleges and universities around the world, with many electing to study in North America. Approximately 98\% of CMIS students attend post­secondary institutions upon graduation. For a list of colleges and universities to which CMIS Class of 2015 graduates were offered acceptance, please see our Current School Profile.

\subsection{College and University Placement}

In recent years, CMIS graduates have been accepted to a wide variety of international colleges and universities, as listed below. In keeping with this tradition of excellence, students in our 2015 graduating class have been offered acceptance into an impressive array of colleges and universities throughout the world. Those universities in which our current graduates have been offered acceptance are indicated with an asterisk

\minor{United States}
\begin{itemize}
\item Baylor University, Waco, TX
\item *Bentley University, Waltham, MA
\item Boston University, Boston, MA
\item Biola University, La Mirada, CA
\item *Bryn Mawr College, Bryn Mawr, PA
\item *California Polytechnic State University, San Luis Obispo, CA
\item California State University Fullerton, Fullerton, CA
\item *Calvin College, Grand Rapids, MI
\item *Carroll College, Waukesha, WI
\item Clark University, Worcester, MA
\item *Cornell University, Ithaca, NY
\item Davidson College, Davidson, NC
\item Emory University, Atlanta, GA
\item Fordham University, Bronx, NY
\item *Grinnell College, Grinnell, IA
\item *Hult, International Business School, Boston, MA
\item *Ithaca College, New York, NY
\item Lancaster Bible College, Lancaster, PA
\item Lafayette College, Easton, PA
\item Lewis and Clark College, Portland, OR
\item Marquette University, Milwaukee, WI
\item *Messiah College, Mechanicsburg, PA
\item Michigan State University, East Lansing, MI
\item Mississippi State University, MS
\item *Mount Holyoke College, South Hadley, MA
\item *New York University, Greenwich Village, NY
\item *Northeastern University, Boston, MA
\item Ohio Wesleyan University, Delaware, OH
\item Penn State University, State College, PA
\item Purdue University, West Lafayette, IN
\item Rochester Institute of Technology, Rochester, NY
\item Rutgers University, New Brunswick, NJ
\item Rice university, Houston, TX
\item Savannah College of Art and Design (SCAD), Savannah, GA
\item State University of New York, (SUNY), Buffalo, NY
\item Syracuse University, Syracuse, NY
\item Texas AandM University, College Station, TX
\item *University of Akron, Akron, OH
\item *University of California, Irvine / Davis / Riverside, CA
\item *University of Connecticut, Storrs, CT
\item *University of Illinois Urbana Champaign, Champaign, IL
\item *University of Massachusetts, Amherst, MA
\item *University of Michigan, Ann Arbor, MI
\item University of San Francisco, San Francisco, CA
\item University of Washington, Seattle, WA
\item Vanderbilt University, Nashville, TN
\item Virginia Polytechnic Institute, Blacksburg, VA
\item *Wheaton College, Wheaton, IL
\end{itemize}

\minor{Canada}

\begin{itemize}
\item Carleton University, Ottawa, Ontario
\item *Trinity Western University, BC
\item University of British Columbia, Vancouver, BC
\item * University of Toronto
\item *University of Waterloo, Toronto
\end{itemize}

\minor{Europe}

\begin{itemize}
\item *Ecole hoteliere de Lausanne, Lausanne, Switzerland
\item *Eindhoven University of Technology, Netherlands
\item Erasmus University College, Rotterdam, Netherlands
\item *Hague University of Applied Science, Netherlands
\item *Hanze University, Groningen, Netherlands
\item University College Roosevelt, Middelburg, Zeeland, Netherlands
\item University of Groningen (RUG), Groningen, Netherlands
\item *University of Aberdeen, Scotland, UK
\item *University of Nottingham, England, UK
\item *University of Sheffield, England, UK
\item *University of Stirling, Scotland, UK
\item *University of Strathclyde, Scotland, UK
\end{itemize}

\minor{Australia}

\begin{itemize}
\item *Le Cordon Bleu, Culinary Arts Institute, Sydney
\item *Blue Mountains, International Hotel Management School, Sydney
\end{itemize}

\minor{Thailand}

\begin{itemize}
\item *Assumption, University, (ABAC), Bangkok
\item *Chulalongkorn University, Bangkok
\item *Mahidol University, Bangkok
\item Payap University, Chiang Mai
\item *Thammasat University, Bangkok
\end{itemize}

\minor{Other Parts of Asia}

\begin{itemize}
\item *Ritsumeikan Asia Pacific University, Oita, Japan
\item *Nanyang Technological University, Singapore
\item *Shanghai Jiao Tong University, Shanghai, China
\item *State University of New York (SUNY), Seoul, Korea
\end{itemize}

\subsection{University Entrance}

\section{Faculty and Staff Data}

CMIS is licensed by the Ministry of Foreign Affairs, owned by the Church of Christ in Thailand (CCT), a national Protestant organization, and operated through a nine-member Board of Directors.  The day-to-day activities of the school are run by a School Management Team (SMT) composed of a Director (Thai), a Superintendent (non-Thai), two Principals (Pre-K - 8 and High School; non-Thai), a School Manager (Thai), and an Assistant School Manager (Thai). 

\begin{description}
\item[Director:]  Manoonvatana Sirisujin         		director@cmis.ac.th
\item[Superintendent:]  Ronelda (Nel) Capadona	superintendent@cmis.ac.th
\item[ES/MS Principal:]  Tyler Stinchcomb	     	elementary@cmis.ac.th
\item[HS Principal:]  Aaron Willette		   	hsprincipal@cmis.ac.th	
\item[Manager:]  Patcharin (Nang) Jingkaojai		manager@cmis.ac.th
\item[Asst. Manager:]  Peay Tananone			assistant\_manager@cmis.ac.th 
\end{description}

\placeholder{Teacher and Admin Qualifications}

\section{Staff Professional Development}

CMIS is committed to providing and supporting professional development opportunities for its full-time teachers to improve teaching and student learning. This commitment is grounded in the belief that professional development is a continuous process, one which may be individualized depending on the skills and needs of the teacher for the purpose of benefiting the teacher, CMIS and its students.  

The quality of our Teachers has been identified as the main factor in attracting new students to CMIS.  Recognizing that, CMIS provides teachers and staff with high quality professional development opportunities, both internally and externally.  The school provides funding in the amount of 10,000 Baht for each school year, for each full-time teacher to engage in professional development opportunities for the purpose of benefiting the teacher, CMIS and its students.  

PD funds can be accrued up to 5 years. Any (projected) unspent amount will be returned to the general PD fund at the end of the 5-year period or earlier if the teacher leaves CMIS employment.

The PD budget may be used toward professional development (PD) recertification, workshops, conferences, professional memberships, or position-specific training.   Travel, per diem, accommodation, registration, tuition, and required materials are also eligible for reimbursement, based on established guidelines.  Transportation budgets are limited, and staff are encouraged to look for PD opportunities within the region.  

All professional development funding and leave must be approved by the divisional principal, superintendent, or director.  CMIS believes that professional development is a continuous process, which may be individualized depending on the skills and needs of the teacher in the support of student learning. Any professional development opportunity should be related to schoolwide, student learning focus. 

All applications for professional development are considered carefully based on the following:
\begin{enumerate}
\item Location of the event with priority set in the order of Chiang Mai, Thailand, SE Asia, outside of SE Asia, as well as if the course or seminar is offered online.
\item Relevance to the school wide, student learning focus.
\item Benefit to CMIS.
\item Length of time away from school.
\item Budget availability (early application is best for this).
\item Timing of request, advanced notice is preferred.
\item Staff member’s seniority or stated plans to stay at CMIS.
\end{enumerate}
	
CMIS promotes the continued professional development of its teachers in the following ways:


\begin{enumerate}
\item Advanced Placement (AP) Training.  
\item Chiang Mai based Conferences and Seminars. 
\item Online courses and seminars.  
\item Conferences and seminars held in Thailand. 
\item EARCOS Conference.
\item Out-of-country conferences and seminars. 
\item Sharing the Experience.  Upon the staff member’s return to school, he or she is expected to share with the school what was learned.  This will be arranged with the Superintendent/school Principal and a time set at an appropriate meeting to share.  Staff members will also file a written assessment of the program through the Professional Development Assessment Form that will be made available for the staff to review.
\end{enumerate}

\placeholder{Professional Develompent Data}

\section{Non-Academic Activities}
\section{School Finance}
\section{Discipline}
\section{Student Health}
\section{Student Service}
\section{Student Performance}
\section{School Community Services}
\section{Faculities Development - School Development Plan}
\section{Parent Teacher Group}
Teachers, administrators and parents or guardians of CMIS students are all automatically members of the Parent Teacher Group (PTG). The PTG is your "voice" at CMIS, as it provides a forum for sharing ideas and concerns about our school, and creates opportunities for you to establish friendships and networks within the CMIS community. The primary objectives of the PTG are:
\begin{itemize}
\item To promote communication between CMIS parents, teachers and administrative staff.
\item To promote and actively support CMIS educational programs, sports, activities and events. 
\end{itemize}
We achieve this through volunteer work and fundraising.  The PTG welcomes and encourages all parents and teachers to take an active role in the school. You may do this by attending our monthly meetings, and by volunteering your time, resources and abilities to make CMIS the best place that it can be - for the sake of our children and their education.

\section{School Executive Team}

The school executive team (SET) is made up of the director, manager, and superintendent. The SET is responsible for implementing the school vision and mission, and serving as a bridge between the Board and the school.  They meet weekly to make high level decisions affecting school policies and procedures. SET, as members of the school board, ??? and raise issues to the school board 
 The current school executive team members are:
\begin{description}
\item[CMIS Director]Manoonvatana Sirisujin
\item[CMIS Manager]Patcharin Jingkaojai
\item[CMIS Superintendent]Ronelda Capadona
\end{description}

\section{School Management Team}
The school management team (SMT) exists to further implement SET decisions.  The SMT is responsible for the day-to-day, administrative, operation of the school.  The SMT would be involved with issues affecting the school facility, curriculum, and student body..  The SMT consists of the SET plus the Assistant Manager, the Elementary/MS Principal, and the High School Principal. 

The current school management team members are:
\begin{description}
\item[CMIS Director] Manoonvatana Sirisujin
\item[CMIS Manager] Patcharin Jingkaojai
\item[CMIS Superintendent]Ronelda Capadona
\item[CMIS Elementary / Middle School Principal] Tyler Stinchcomb
\item[CMIS High School Principal] Aaron Willette
\item[Assistant Manage] Peay Tananone
\end{description}

\section{School Board}

CMIS is governed by a Board of nine Directors, four of whom are appointed by the Foundation of the Church of Christ in Thailand (CCT);  the School Manager, Director, Superintendent; as well as two who are elected from the CMIS community, which include one teacher representative and one PTG representative. 

The current school board members are: 
\begin{description}
\item[Chair] Rev. Dr. Esther Wakeman 
\item[Secretary] Rev. Dr. Sharon Bryant
\item[Board Member] Dr. David Filbeck \debug{What are Filbeck and Mcdaniel's  titles?}
\item[Board Member] Kathryn Mcdaniel
\item[CMIS Director] Manoonvatana Sirisujin 
\item[CMIS Manager] Patcharin Jingkaojai 
\item[CMIS Superintendent] Ronelda Capadona 
\item[PTG Representative] Pascal van Geest
\item[Teacher Representative] Brad Schmock
\end{description}
The Roles and Responsibilities of the Board are listed below:  (listed on page 9 of the Student Planner).
CMIS is owned by the Foundation of the Church of Christ in Thailand (CCT), and is operated through a Board of Directors comprised of a Board Chair (appointed by CCT); the CMIS School Executive Team; an elected representative from the PTG; an elected representative from the teaching staff; and additional members appointed by the CCT.

\minor{Strategic Planning and Thinking}
The Board of Directors develops and maintains the strategic plan for the school, guiding
decisions on the organizational level in terms of program, facilities, etc., while keeping in
mind the overall vision and mission of the school.

\minor{Setting Policy}
The Board of Directors oversees the development of policy for school operations. Hiring, evaluating and supporting of the School Leadership Team (Director, Manager and Superintendent) is a key responsibility of the Board. The Board of Directors is not involved in the day-to-day operations of the school, but supports the school leadership in developing the necessary skills and resources to run the school effectively.

\minor{Financial Stability}
The Board of Directors is responsible for the financial stability of the school and thus sets tuition rates and approves annual budgets.

\minor{How Does the Board Govern?}
The Board of Directors generally meets once a month from July through June.  Board members should be committed to attending every board meeting, but the board is also understanding of other constraints of members’ time.

\section{Teacher Administration Communication Team}

The purpose of the TACT is to foster and create the best learning environment possible for our students by improving the overall happiness and satisfaction of faculty. We believe that when we invest in the well being of people, we invest in the long term success and viability of CMIS.

To this end, the role of TACT is to provide a way for faculty to raise issues, concerns, and questions to the CMIS Executive Team (Director, Superintendent, Business Manager).  CMIS prefers that individuals address issues, concerns, and questions directly to their building administration (the principals of the elementary, middle, and high schools).  There are times, however, when individuals may not feel comfortable in speaking with building administration or leadership; thus TACT can forward these issues on their behalf.
\begin{itemize}
\item we believe in positive intentions, and that all stakeholders share the same vision of the best possible learning environment for CMIS students
\item we believe in openness and transparency in decision making and communications
\item we believe in equity - the fair and impartial treatment of others
\item we believe that praise and recognition produces far better results than criticism or punishment
\item we believe in social sustainability and social responsibility
\end{itemize}

The operation of TACT follows these policies, procedures, guidelines:

\minor{Privacy Policy}
\begin{itemize}
\item the identity of the person(s) bringing forth issues/questions/concerns is kept confidential, unless the person(s) so inform TACT they would like their name(s) to be known
\item the minutes and all other communications published by TACT are only to be shared with CMIS stakeholders (CMIS leadership, faculty, Board of Directors)
\item faculty are not required to put their name(s) on communications with TACT - though this may make it difficult for TACT to ask for clarification and lead to an issue/concern/question not being raised (clarification is often required and is confidential)
\end{itemize}

\minor{Meetings}
\begin{itemize}
\item TACT representatives endeavor to meet regularly, from September - May, with CMIS Executive Team (Director, Superintendent, Business Manager)
\item TACT representatives will plan and coordinate the date of the monthly meeting, usually for the last week of each month in the Superintendent's office
\end{itemize}

\minor{Communications Procedures and Policies}
\begin{itemize}
\item TACT will publish the minutes of the monthly meetings with leadership by informing faculty, leadership, and the Board of Directors of these minutes through email
\item TACT will inform faculty, at least one week in advance,  of any upcoming meetings with CMIS leadership and request input from faculty
\end{itemize}

\minor{Issues/Questions/Concerns (guidelines)}
\begin{itemize}
\item the issues/questions/concerns must be related to CMIS faculty (those holding a teacher's contract with CMIS)
\item faculty are encouraged to first raise issues/questions/concerns, where relevant, with their respective building administration, or with CMIS leadership directly, before requesting TACT assistance
\item the issues/questions/concerns must have an importance/relevance to many faculty
\item some issues/questions/concerns may not be forwarded by TACT if they have already been responded to by leadership
\item TACT questions, discussions and minutes must be issue-based.  Person-specific questions raised and publicly minuted in TACT notes are inappropriate and potentially hurtful. Issues/questions/concerns should be researched and understood and the goal would be to find solutions that benefit CMIS as an institution and learning environment.
\item issues/questions/concerns can be brought to TACT through these methods:
\begin{enumerate}
\item speaking with TACT rep for your division
\item dropping a note in the mailbox of your TACT rep
\item email your TACT rep directly
\end{enumerate}
\end{itemize}

\minor{Grievances}

When a faculty member has a grievance, either with another faculty member, leadership, etc. TACT can help facilitate the mediation process. There is an existing CMIS procedure and explanation for grievances detailed in the CMIS Faculty Handbook.

TACT representatives are familiar with the CMIS grievance procedure, and can help you by making sure these steps are followed. TACT reps role is not to participate in the procedure, rather just help to ensure the process is followed. This applies to both the informal and the formal procedures as outlined in the CMIS Faculty Handbook.

\minor{TACT Representatives - policies and procedures}
\begin{enumerate}
\item there are three TACT representatives, one at each division (elementary, middle, high)
\item before the end of May of each year, nominations are requested from the faculty in each division for the representative for that division for the following school year
\item for consistency and to carry forward knowledge, at least one member of  TACT must continue from one year to the next, thus nominations may not occur in all three divisions
\item the representatives from the previous year will oversee the nomination process for their respective divisions
\item the nomination process uses the following text, and can be run through email, at a faculty meeting, by paper ballot, a google docs form, or any other reliable and equitable method:
"I nominate\underline{          }the Teacher/Admin Communication Team representative for the  (division name) for the\underline{          }school year"
\end{enumerate}

\section{Student Council}

\placeholder{Student Council data}

CMIS Student Council (StuCo) is a representative structure for all the students in MS and HS.  It provides students with the opportunity to become involved in the affairs of the school, working in partnership with school management, staff, community, and parents.  It should always work for the benefit of the school and its students.  

\begin{itemize}
\item Communicating and consulting with students in the school
\item Involving as many students as possible in the activities of the council and school
\item Planning and managing the council's programme of activities for the year with reviews and evaluations to improve upon and change when necessary
\item Communication and co-operation with school staff, board and management
\item Working with parent's council (booster club) in school
\item Involvement in school planning
\item Having a say in school awareness and policies e.g. anti-bullying, homework, substance use, mobiles, healthy eating, code of discipline, uniform, etc
\end{itemize}

\section{Conclusion}

\chapter{Progress Report}
\chapter[Student/Community Profile Summary]{Student/Community Profile - Overall Summary from Analysis of Profile Data and Progress}
\chapter{Self-Study Findings}
\section{Orgianization for Student Learning}
\section{Curriculum, Instruction, and Assessment}
\subsection{Current Educational Research and Thinking}
\indicator{The comprehensive and sequential documented curriculum is modified as needed to address current educational research and thinking, other relevant international/national/community issues and the needs of all students.}

\prompt{Comment on the effective use of current educational research related to the curricular areas in order to maintain a viable, meaningful instructional program for students. Examine the effectiveness of how the school staff stay current and relevant and revise the curriculum appropriately within the curricular review cycle.}

\begin{findings}
CMIS Leadership and Teaching Staff use curriculum that is comprehensive and sequentially documented  that can be modified as needed to address current educational research and thinking, relevant international/community issues, and the needs of all students.

CMIS Leadership Team strongly believes that at the core of a rigorous, engaging, coherent curriculum are research-based standards. The  CMIS Teaching Staff use multiple, comprehensive, and appropriately sequenced standards that inform and provide the foundation of our curricular decisions (see section entitled, Academic Standards in Each Area). 

The 2013-2014 academic year saw a great deal of change to the standards (which impact curricular decisions) as the CMIS Teaching Staff, with the support of the Leadership Team adopted new standards in ELA, mathematics, and science, as well as finalizing the adoption of all other content standards (i.e. physical education, social studies, health, fine arts, world languages). 

The adoption of these standards, especially in ELA, mathematics, science, and 9-12 history  reflect significant shifts in conceptual understanding of the content, have major implications on instruction and assessment, and illuminate real increases in depth of knowledge and rigor. All curricular decisions would have to be made based upon these new realities. Furthermore, CMIS has begun to view our adopted standards as the bedrock of all unit creation and planning. Standards also continue to help guide appropriate instruction and ensure rigor. 

Because of these conceptual shifts, the 2014-2015 saw a strong focus on professional development topics relating to the difference between curriculum and standards, and the shared responsibility of the CMIS Literacy Standards. Teachers, community members, and administrators were involved in multiple discussions and the CMIS Leadership Team believed this was an appropriate starting point for teachers and community members to begin understanding the standards and curriculum.
\end{findings}

\begin{evidence}
\item PTG Curriculum, Professional Development Folder \url{https://drive.google.com/a/cmis.ac.th/folderview?id=0ByVFfrm0zfolWE0yenprdktGVlk&usp=sharing}
\end{evidence}

\begin{findings}
Because of the adoption of the CMIS Literacy Standards, a great deal of training during the 2014-2015 school year focused on the ELA standards. Currently, all curriculum decisions, including resource/curriculum purchasing, must adhere to our standards with fidelity. All reading and writing curriculum decisions are based on ensuring appropriate complexity, balancing informational text, reading in the disciplines, and academic vocabulary (for reading). Writing from sources and appropriate text types are required for writing curriculum decisions. 
\end{findings}

\begin{evidence}
\item CMIS Understanding by Design Resources \url{https://drive.google.com/a/cmis.ac.th/folderview?id=0ByVFfrm0zfolfmUyZV9DbGoxZVhpVHpGdG9MeEt6MHZJaEtoT3VzTjM0bkk5NFQ5MVJldUU&usp=sharing}
\item CMIS Understanding By Design Master List
\end{evidence}

\begin{findings}

Since curricular decisions should be made using clear standards and student outcomes as a guide, a new Curriculum Review Cycle/Resource Adoption strategic plan had to be developed and implemented. The most important elements of the Curriculum Review process are the specific evaluations tools, rubrics, and vetting instruments that have been developed and used to narrow down vendors and products for curricular and resource materials. CMIS Leadership also provides the necessary time and space required for teacher collaboration in vetting and evaluation of curricular items. 

Science curriculum was the first content to be vetted and evaluated based upon the Curriculum Review Cycle; science curriculum was purchased during the 2015-2016 school year. As with ELA curriculum decisions, science saw significant conceptual changes with the adoption of the the Next Generation Science Standards. Because of this, CMIS Leadership, developed vetting and evaluation instruments to ensure curriculum and resource alignment with the standards, as well as rigor and student engagement. This year, Mathematics undertook the same curriculum vetting process and ELA/social studies will be reviewed next year. 

In order to ensure that the resources we purchase outside of the normal adoption cycle are of the highest quality and they meet our three general criteria of: is it aligned to the standards? Is it rigorous? and is it engaging? Resource Request and Resource Renewal forms were developed and implemented. 
\end{findings}

\begin{evidence}
\item "Teacher Dashboard." CMIS Curriculum. Chiang Mai International School, 22 July 2016. Web. 22 July 2016.
\end{evidence}

\begin{findings}
Science curriculum was the first content to be vetted and evaluated based upon the Curriculum Review Cycle; science curriculum was purchased during the 2015-2016 school year. As with ELA curriculum decisions, science saw significant conceptual changes with the adoption of the the Next Generation Science Standards. Because of this, CMIS Leadership, developed vetting and evaluation instruments to ensure curriculum and resource alignment with the standards, as well as rigor and student engagement. This year, Mathematics undertook the same curriculum vetting process and ELA/social studies will be reviewed next year. 

In order to ensure that the resources we purchase outside of the normal adoption cycle are of the highest quality and they meet our three general criteria of: is it aligned to the standards? Is it rigorous? and is it engaging? Resource Request and Resource Renewal forms were developed and implemented. 
\end{findings}

\begin{evidence}
\item "Model Content Frameworks." Model Content Frameworks. PARTNERSHIP FOR ASSESSMENT OF READINESS FOR COLLEGE AND CAREERS, 2016. Web. 20 July 2016. \url{http://www.parcconline.org/resources/educator-resources/model-content-frameworks}
\end{evidence}

\begin{findings}
The CMIS Google Drive is the primary method of organizing and archiving  of the curricular units. The Drive along with the the CMIS Teacher Dashboard, found on the school website, is a place where teachers can access these resources. The Dashboard also contains the CMIS Standards Blueprints. Teacher feedback and anecdotal evidence suggested that greater alignment of the UbD units to the standards would be possible if the teachers had a general blueprint or pacing guide to help them visualize and map the whole year of standards. Department and grade level teams began the blueprint work at the beginning of the 2015-2016 year. All core subjects, with the exception of K-5 ELA and mathematics, have been completed. 
\end{findings}

\begin{evidence}
\item Resource Request Form \url{https://docs.google.com/a/cmis.ac.th/forms/d/1hBKsxgHpPSOT0MtUTsZBoVX2t3ztxeGVY4R4BVNeVL8/edit?usp=sharing}
\item Resource Renewal Form \url{https://docs.google.com/a/cmis.ac.th/forms/d/1i6EnhitZ2yJZh-1OjL8jsJ2pFa7oRJCP0VcA50b_D6g/edit?usp=sharing}
\end{evidence}

\begin{findings}
ELA blueprints posed a challenge as standards are addressed all year long. In order to address this challenge, small groups were scheduled for focused professional development using the Partnership for Assessment of Readiness for College and Careers (PARCC) ELA Model Frameworks. Collaboration time was given to ELA teachers in grades 6-12 to complete this work. The CMIS Professional Development team has scheduled these same professional development opportunities for K-5 during the 2016-2017 school year. 

The combined teamwork of all stakeholders: the CMIS Leadership Team, the community, outside research, vetted educational organizations, and, most importantly,  the teachers themselves, ensure that CMIS staff at all levels stay current in educational research.
\end{findings}

\subsection{Academic Standards for Each Area}

\indicator{The school provides a comprehensive and sequential documented curriculum that is articulated within and across grade levels for the improvement of programs, learning, and teaching.}

\prompt{Evaluate to what extent there are defined academic standards for each subject area, course, and/or program (e.g., online instruction) that meet state or national/international standards.}

\begin{findings}
CMIS Leadership and Teaching Staff strongly believe in the importance of the school’s academic standards. CMIS Staff use a common understanding of academic standards: 

Academic standards define the concepts, skills, and knowledge that students should know and be able to do in each curricular area, the level at which students are expected to demonstrate this knowledge, and grade-level expectations for performance. In a standards-based educational system, schools determine the benchmarks for student work that meet these standards, provide appropriate instruction, and use multiple assessment measures to identify the level of achievement for all students. This approach assists the schools in defining the quality accomplishment of the complementary schoolwide learner outcomes and the degree to which all students are achieving them. Standards do not describe any particular teaching practice, curriculum, or assessment method 
\end{findings}

\begin{evidence}
\item Abbot, S. "Learning Standards Definition." The Glossary of Education Reform. The Great Schools Partnership, 2013. Web. 21 June 2016. \url{http://edglossary.org/learning-standards/}
\end{evidence}

\begin{findings}
CMIS provides a comprehensive and sequential documented curriculum that is articulated within and across grade levels for the improvement of programs, learning, and teaching.

CMIS teachers and students use multiple research-based standards for all grade levels and content areas. The CMIS Leadership and Professional Development Team use teacher feedback to address possible changes or modifications to the standards. 

Adopted CMIS standards include:
Common Core State Standards ELA*
Common Core State Standards Mathematics*
Next Generation Science Standards (NGSS)*
AERO Standards (aligned to NCSS framework) grades K-8*
National Standards for History (from UCLA) grades 9-12
C3 Framework (piloted 2016-2017)*
AP Course Topics and Indicators
National Coalition for Core Arts Standards  (NCCAS)
Computer Science Teachers Association Framework (CSTA)
National Educational Technology Standards for Students (NETS)
World Readiness Standards for Learning Languages
Physical Education Model Content Standards for California Public Schools 
National Health Education Standards (Center for Disease Control)
Standards for the 21st Century Learner (American Association of School Librarians)
Wisconsin’s Model Academic Standards for Business  

* standards are internationally benchmarked 

Research indicates CCSS (both ELA and Mathematics), NGSS, and C3 are standards that are considered rigorous, world class, and allow for deeper engagement around fewer concepts/topics.
\end{findings}

\begin{evidence}
\item CCSS-FAQ, Setting Criteria, Myth vs. Fact,Research Supporting Key Elements of the Standards: Appendix A. \url{https://drive.google.com/a/cmis.ac.th/file/d/0ByVFfrm0zfolcjRFZWtWWUhvR0U/view?usp=sharing}
\item NGSS-Standards Background:  Research and Reports \url{http://www.nextgenscience.org/standards-background-research-and-reports}
\item C3-Appendix C: Scholarly Rationale for the C3 Framework (p.74) \url{https://drive.google.com/a/cmis.ac.th/file/d/0ByVFfrm0zfolVGRyN0VKZzVlLVE/view?usp=sharing}
\item NCCAS-A Conceptual Framework for Arts Learning (p. 23) \url{http://www.nationalartsstandards.org/sites/default/files/NCCAS\%20\%20Conceptual\%20Framework_0.pdf}
\end{evidence}

\begin{findings}
In order to address the prior visiting team observation that some teachers did not feel that they understood the standards, CMIS has provided professional development for the CCSS ELA  and NGSS. Starting from the critical conceptual shift found in the CCSS that literacy is a “shared responsibility” of ALL staff, K-12, all contents, the CMIS Leadership Team devoted a majority of 2014-2015 to unpacking and analyzing the newly adopted ELA standards with the whole staff.  Since ALL teachers have literacy standards, the professional development ensured that teachers not only focused on their own grade-level standards (i.e “...within the grade level”), but also worked with the anchor standards that addressed vertical progression of outcomes (i.e. “...across grade levels”). 

UbD plans, mid/end of year assessment, peer reflection, and administrative observational walkthroughs are all methods to ensure instructional fidelity to the standards. 

As mentioned earlier, CMIS Leadership Team believe that all curricular decisions should be based on three basic questions: is it aligned to the standards, is it rigorous, and is it engaging? During the 2015-2016 school, the CMIS Middle and High School Science department, representative teachers from K-5, and CMIS Leadership team vetted and evaluated science materials using CMIS developed rubrics and instruments based upon EQuIP (Educators Evaluating the Quality of Instructional Products) Rubric. The outcome was the purchase of FOSS Kit science materials that met and exceeded our criteria for grades K-5. FOSS Kits for science have been purchased and CMIS will continue to provide time and assistance to help teachers utilize and implement this curriculum resource. 

This year, mathematics curriculum was reviewed and and resources were vetted. Through the use of Instructional Materials Evaluation Tool (IMET) developed by Achieve the Core, the K-5 group adopted two resources that aligned strongly with the standards and the instructional shifts in mathematics (Focus, Coherence, and Rigor). The first resource from Great Minds Inc. is entitled Eureka Math. Eureka Math not only aligns with the standards, it provides appropriate coherence, focus, and rigor. The second resource vetted and purchased was My Math published by McGraw Hill. Again, the adoption committee  found good alignment to the standards and the instructional shifts. Both programs provide a balance of rigor- application and conceptual knowledge in Eureka and procedural skills in My Math. 

Curricular decisions are also made as a result of our professional development.  After CMIS workshops on the CCSS conceptual shifts and with discussions with teachers, it was determined that the school lacked appropriately complex informational text in grades 9-12. Because of this discovery, plans and funds were set aside to purchase informational text. 

AP courses have appropriately aligned (in rigor and complexity) textbooks and resources.  

Course and grade level blueprints have also been developed to use as a guide for vertical and horizontal integration discussion and collaboration to determine if modification is necessary. Areas that have been discussed are standards overlap, skill/content gaps
\end{findings}

\subsection{Embedded Global Perspectives}
\indicator{The school leadership and certificated staff ensure that global education concepts, perspectives, and issues are embedded within the curricular areas.}

\prompt{Examine the curricular documentation and observe the delivered curriculum to determine the extent to which there is integration of global concepts, perspectives, and issues.}

\begin{findings}
The CMIS Leadership and Teaching Staff ensure that global education concepts, perspectives, and issues are embedded within the curricular areas.

CMIS Leadership and Teaching Staff wanted to describe evidence that goes beyond the traditional list of events that international schools normally cite for embedding global perspectives. Though CMIS Leadership, teaching staff, and community still organize and implement an International Day which rotates with Thai Day (i.e. every other year); other major events events such as Model United Nations and National History Day are also scheduled. Each of these events showcase and celebrate our unique place on the globe and our diverse community. There is also evidence of more routine and thoughtful examples of how CMIS promotes and embeds global perspectives. 

Most importantly, and generally overlooked, are our adopted standards. Of the core courses (ELA, Mathematics, Social Studies, and Science), all four set of standards are internationally benchmarked. This benchmarking  ensures CMIS students are exposed to global perspectives. As Achieve (2014) states about our adopted ELA and Mathematics standards, 
“As part of the Common Core State Standards Initiative, Achieve helped collect and analyze standards from a number of countries. These studies helped inform the choices made by the writers of the common standards.”
For Next Generation Science Standards, Achieve stated: 
“The overall goal of Achieve’s study on international standards is to inform the development of the NRC framework and next-generation science standards.”
And finally, College, Careers, and Civics Framework (C3) states: 
“...standards (including C3) suggests that all standards should be rigorous, world class, and internationally benchmarked, while also allowing for deeper engagement around fewer concepts/topics.” 
CMIS Leadership feel confident that as we remain focused on aligning appropriate standards to our assessments, lessons, and instruction, we can be assured that we are providing standards with a global perspective 
As illustrated in other sections, standards play a central role in planning, guiding, and developing curriculum. 
The counseling and student support departments provide multiple, targeted lessons on global perspectives. Units such as…
The diversity of our Leadership Team and Teaching Staff ensure multiple perspectives, concepts, and issues are are embedded in CMIS lessons and discussions. Diversity of not only nationality, but age, gender, and experiences makes CMIS a dynamic and engaging global workplace. 

The library provides books on multiple global issues and from a wide variety of perspectives and diverse nationalities.
\end{findings}

\begin{evidence}
\item "International benchmarking is important from a national perspective to ensure our long-term economic competitiveness." Achieve. 2012. Web. 06 May 2016. \url{http://www.achieve.org/international-benchmarking}
\item "International Science Benchmarking Report Taking the Lead in Science Education: Forging Next-Generation Science Standards Executive Summary."International Science Benchmarking Report. Achieve, 2010. Web. 20 May 2016. \url{http://www.achieve.org/files/InternationalScienceBenchmarkingReportExecutiveSummary.pdf}
\item USA. Kentucky Department of Education. Office of Next Generation Learners.704 KAR 3:303, Kentucky Core Academic Standards (First Reading). Web \url{https://drive.google.com/a/cmis.ac.th/file/d/0ByVFfrm0zfolNWh1X2w5NV9uNlU/view?usp=sharing}
\end{evidence}

\subsection{Congruence}
\indicator{There is congruence between the actual concepts and skills taught, the academic standards, and the schoolwide learner outcomes.}

\prompt{Evaluate if there is congruence between the actual concepts and skills taught, the academic standards, and the schoolwide learner outcomes.}

\begin{findings}
CMIS Leadership and teaching staff ensure congruence between the actual concepts and skills taught, the academic standards, and the schoolwide learner outcomes through a variety of methods. 

Much of the professional development and teacher work for the past two years has focused on unpacking the literacy standards as all teachers are responsible student success in reading, writing, and speaking/listening. As we focused on understanding the standards, certain structures were created to ensure fidelity to concepts/skills, standards, and SLOs. 

First, was the continuation of the Understanding by Design unit development initiative begun in 2013-2014. The traditional UbD template ensures that teacher/designers address in Stage 1: established goals, skills and knowledge, essential questions, and understandings. Stage 2: assessment, and Stage 3: learning plan. Though CMIS teachers used this general UbD template to ensure alignment with these essential, researched-based elements of UbD, a handful of modifications were added to the template to ensure congruence between our skills/knowledge, our adopted standards, and the SLOs; see Table \ref{table:tab1} below. 
%\begin{comment}

\begin{tbl}{\textwidth}{XXX}{test caption}{table:tab1}%{\captionof{table}{UbD Unit Revisions Since 2015 to Address Congruency}\label{table:tab1}}
\toprule[3px]
Established Goals &
Separate out Primary and Secondary Standards &
CCSS/NGSS/C3 were created to be integrated. K-12 focus on creating rich learning experiences that address multiple standards \\
\midrule
Assessment/Learning Plan &
Formative Assessment Section & 
Formative assessment was an integral part of professional development. Section created to provide place for teachers to practice. \\
\midrule
Learning Plan &
Learning Activity/ 
Experience Alignment &
Teachers were asked to explicitly align the standards to the learning experience. If not possible, they were asked why that activity was important.  \\
\midrule
Learning Plan &
Close Reading &
Anchor standard 1 asks students to “read closely”. Close reading is an outcome important to students in grades K-12. \\ 
\midrule
Learning Plan &
Complexity Level &
Shift 3 of CCSS ELA requires a “staircase of complexity”. Appropriate complex text for each grade level is essential to complete this shift.  \\
\midrule
Topic Section &
SLO &
Teachers and Leadership felt strongly that the most effective way to teach SLOs was implicitly, through the UbD units. \\
\bottomrule
\end{tbl}


%\end{comment}

Secondly, CMIS Leadership and Teaching Staff have implemented the Datawise Process. Using this process, CMIS has created school data teams of teachers and administrators who make use of performance data and other information to target educational questions to pursue, identify major gaps in student understanding, identify target areas called learner-centered problems (LCP), reframe learner-centered problems as problems of practice (POP), target solutions to problems of practice, and write action plans pinpointing how broadly solutions will be implemented, how they can be implemented and adapted across grade levels and content areas, and on what timelines they will be carried out. 

CMIS Leadership and Teaching Staff used 2014-2015 ISA data to develop a CMIS specific schoolwide LCP, created a schoolwide POP, each department created a strategy to address the POP, and assessed throughout the 2015-2016. Based upon teacher feedback and achievement data, the 2016-2017 year Datawise program was modified to allow each department to determine their LCP and POP based upon department specific data. The departments’ Datawise plan is currently being implemented. By using the Datawise process, the CMIS Leadership the Teaching Staff have ensured academic outcomes (i.e. standardized test data) are used to make instructional decisions that were based upon students’ skills/ concepts, and standards. 

Looking at student work

Review the SLOs during teacher orientation (2016-2017) and plan implementation for review with students 
\end{findings}

\begin{evidence}
\item McTighe, Jay, and Elliot Seif. "A Summary of Underlying Theory and Research Base for Understanding by Design by Jay McTighe and Elliott Seif." A Summary of Underlying Theory and Research Base for Understanding by Design by Jay McTighe and Elliott Seif. Jay McTighe Educational Associates, 2002. Web. 4 Apr. 2016. \url{http://jaymctighe.com/wordpress/wp-content/uploads/2011/04/UbD-Research-Base.pdf}
\item Nel has photos
\item Houdett, Kathryn Parker, Elizabeth A. City, and Richard J. Murnane. Data Wise. Cambridge: Harvard Education, 2013. Print. \url{http://isites.harvard.edu/icb/icb.do?keyword=datawise&pageid=icb.page555331}
\end{evidence}

\subsection{Student Work--Engagement in Learning}
\indicator{The school’s examination of representative samples of student work and snapshots of student engagement in learning demonstrates the implementation of a standards-based curriculum and the schoolwide learner outcomes.}

\prompt{Evaluate to what extent the examination of representative samples of student work and snapshots of student engagement in learning demonstrate the implementation of a standards-based curriculum and the addressing of the schoolwide learner outcomes.}

\begin{findings}
CMIS Leadership and Teaching Staff engage in the examination of representative samples of student work and snapshots of student engagement in learning that demonstrates the implementation of a standards-based curriculum and the schoolwide learner outcomes.

CMIS Teachers continue to use student work to make informed instructional decisions. CMIS Leadership has implemented multiple research-based instruments to examine student work to ensure  alignment to student learner outcomes and CMIS standards; also to ensure effective collaboration. Professional discussions about looking at student work  balance professional development about how and why looking at student work is important and the opportunity to examine student work. CMIS continues to discuss the importance of looking at data critically, without judgement, over interpretation, keeping it private, or ”taking it” personally. To address this need, research journals were read and discussed. 

CMIS Teaching Staff have used the adapted Longfellow Slice as well as CMIS teacher created protocol-called the CMIS Slice (in development). CMIS Leadership uses norms from National School Reform Faculty and Annenberg Learner’s Critical Friends Group. Student Learner Outcomes were also examined in context of examining student work. Data from these discussions indicates that student learner outcomes could be more explicitly addressed in student work. Also, a teacher and leadership discussion suggests that CMIS teachers and students would benefit from utilizing one examining student work protocol to be consistent. 

The Student Support department consistently use student samples to assist students in meeting or exceeding the academic standards. 
\end{findings}

\begin{evidence}
\item Some Guidelines for Learning From Student Work Adapted from HORACE, November 1996, p.2. 
\item Purpose of the Adapted Longfellow Slice (aka the CMIS Slice)
\item Swanson, Kristen, Allen Gayle, and Mancabelli Rob. "Eliminating the Blame Game." Educational Leadership 73.3 (2015): 68-71. ASCD.org. Web. 3 Mar. 2016.
\item MacDonald, Elisa. "WHEN NICE WON’T SUFFICE Honest Discourse Is Key to Shifting School Culture." Journal of Staff Development 32.3 (2011): 45-51. Learning Forward. Learning Forward. Web. 22 Nov. 2015.
\end{evidence}
\section{Support for Student and Personal Academic Growth}
\section{Resource Management and Development}
\subsection{Congruence}

\indicator{There is congruence between the actual concepts and skills taught,the academic standards, and the schoolwide learner outcomes.}

\prompt{Evaluate if there is congruence between the actual concepts and skills taught, the academic standards, and the schoolwide learner outcomes.}

\begin{findings}

CMIS Leadership and teaching staff ensure congruence between the actual concepts and skills taught, the academic standards, and the schoolwide learner outcomes through a variety of methods. 

Much of the professional development and teacher work for the past two years has focused on unpacking the literacy standards as all teachers are responsible student success in reading, writing, and speaking/listening. As we focused on understanding the standards, certain structures were created to ensure fidelity to concepts/skills, standards, and SLOs.

First, was the continuation of the Understanding by Design unit development initiative begun in 2013-2014. The traditional UbD template ensures that teacher/designers address in Stage 1: established goals, skills and knowledge, essential questions, and understandings. Stage 2: assessment, and Stage 3: learning plan. Though CMIS teachers used this general UbD template to ensure alignment with these essential, researched-based elements of UbD, a handful of modifications were added to the template to ensure congruence between our skills/knowledge, our adopted standards, and the SLOs; see Table below. 

Secondly, CMIS Leadership and Teaching Staff have implemented the Datawise Process. Using this process, CMIS has created school data teams of teachers and administrators who make use of performance data and other information to target educational questions to pursue, identify major gaps in student understanding, identify target areas called learner-centered problems (LCP), reframe learner-centered problems as problems of practice (POP), target solutions to problems of practice, and write action plans pinpointing how broadly solutions will be implemented, how they can be implemented and adapted across grade levels and content areas, and on what timelines they will be carried out. 

CMIS Leadership and Teaching Staff used 2014-2015 ISA data to develop a CMIS specific schoolwide LCP, created a schoolwide POP, each department created a strategy to address the POP, and assessed throughout the 2015-2016. Based upon teacher feedback and achievement data, the 2016-2017 year Datawise program was modified to allow each department to determine their LCP and POP based upon department specific data. The departments' Datawise plan is currently being implemented. By using the Datawise process, the CMIS Leadership the Teaching Staff have ensured academic outcomes (i.e. standardized test data) are used to make instructional decisions that were based upon students' skills/ concepts, and standards. 

Looking at student work

Review the SLOs during teacher orientation (2016-2017) and plan implementation for review with students 

\end{findings}

\begin{evidence}
\item McTighe, Jay, and Elliot Seif. "A Summary of Underlying Theory and Research Base for Understanding by Design by Jay McTighe and Elliott Seif." A Summary of Underlying Theory and Research Base for Understanding by Design by Jay McTighe and Elliott Seif. Jay McTighe Educational Associates, 2002. Web. 4 Apr. 2016. 
\item Nel has photos
\item Houdett, Kathryn Parker, Elizabeth A. City, and Richard J. Murnane. Data Wise. Cambridge: Harvard Education, 2013. Print.
\end{evidence}

\chapter{Schoolwide Action Plan}
\chapter{Appendix}
\listoffigures
\listoftables
\end{document}
