\documentclass{report}
\title{WASC Self-Study Report}
\date{November 16, 2016}
\author{CMIS}
\usepackage{tabularx,colortbl,array}
\usepackage[table]{xcolor}
\usepackage{graphicx}
%\usepackage{paracol}
\usepackage{blindtext}
%\usepackage{capt-of}
\usepackage{subfig}
\usepackage{enumitem}
\usepackage{verbatim}

\usepackage{float}
\usepackage{wrapfig}
\floatstyle{boxed} 
\restylefloat{figure}
\usepackage{fancyhdr}
\pagestyle{fancy}
\fancyhf{}
\rfoot{Page \thepage}
\cfoot{\leftmark}
\usepackage{xcolor}
\usepackage{hyperref}
\hypersetup{%
  pdftitle=WASC Self Report,
  colorlinks=true,
  urlcolor=cyan,
  urlbordercolor=magenta
}
\setlength{\parskip}{1em}

\makeatletter
\ifnum\inputlineno=\m@ne
\let\showlineno\@empty
\else
\def\showlineno{ line \the\inputlineno}
\fi
\makeatother

\usepackage[most]{tcolorbox}
\usepackage{marginnote}
\usepackage{booktabs}
\newcommand{\debug}{%
        \marginnote{%
                \begin{debugBox}%
                        \jobname\showlineno%
                \end{debugBox}%
        }%
}

\newtcolorbox{evidenceBox}{%
  colback=black!5,
  colbacktitle=violet,
  boxrule=1pt,
  top=1em,
  breakable,
 % arc=4pt,
 % outer arc=4pt,
  title=Evidence
}

\newtcolorbox{findingsBox}{%
  colback=white,%black!5,
  colbacktitle=red,
  boxrule=1pt,
 % boxsep=1pt,
 % top=1em,%-0.5\baselineskip,
  breakable,
 % arc=4pt,
 % outer arc=4pt,
  title=Findings
}

\newtcolorbox{debugBox}{%
        colback=yellow,
        title=DEBUG
}
        

\newenvironment{findings}
{
\debug
\noindent\textbf{Findings}

}
{
%\end{findingsBox}
}

\newenvironment{evidence}
{
\debug
\begin{evidenceBox}
\vspace{-\topsep}
\begin{itemize}[leftmargin=*]
  \setlength{\parskip}{0pt}
  \setlength{\itemsep}{1pt}
}
{
\end{itemize}
\end{evidenceBox}
%\vspace{4em}
}


\newcommand{\indicator}[1]{ 
\noindent\textbf{Indicator:} #1
}

\newcommand{\prompt}[1]{ 
\noindent\textbf{Prompt:} \textit{#1}
}

\newenvironment{tbl}[2]
{
%\rowcolors{2}{lightgray}{white}
\tabularx{#1}{#2}
%\rowcolor{darkgray}
}
{
\endtabularx
}

\begin{document}
\maketitle

\tableofcontents
\chapter{Preface}

\begin{tbl}{\textwidth}{|X|X|X|}
\textbf{\color{white}this} & \textbf{\color{white}that}& \textbf{\color{white}the other thing} \\
this&that&the other thing\\1&2&3\\a&b&c\\yes&no&maybe\\
\hline
\end{tbl}


\chapter{Student/Community Profile and Supporting Data and Findings}
\chapter{Progress Report}
\chapter[Student/Community Profile Summary]{Student/Community Profile - Overall Summary from Analysis of Profile Data and Progress}
\chapter{Self-Study Findings}
\section{Orgianization for Student Learning}
\section{Curriculum, Instruction, and Assessment}
\subsection{Current Educational Research and Thinking}
\indicator{The comprehensive and sequential documented curriculum is modified as needed to address current educational research and thinking, other relevant international/national/community issues and the needs of all students.}

\prompt{Comment on the effective use of current educational research related to the curricular areas in order to maintain a viable, meaningful instructional program for students. Examine the effectiveness of how the school staff stay current and relevant and revise the curriculum appropriately within the curricular review cycle.}

\begin{findings}
CMIS Leadership and Teaching Staff use curriculum that is comprehensive and sequentially documented  that can be modified as needed to address current educational research and thinking, relevant international/community issues, and the needs of all students.

CMIS Leadership Team strongly believes that at the core of a rigorous, engaging, coherent curriculum are research-based standards. The  CMIS Teaching Staff use multiple, comprehensive, and appropriately sequenced standards that inform and provide the foundation of our curricular decisions (see section entitled, Academic Standards in Each Area). 

The 2013-2014 academic year saw a great deal of change to the standards (which impact curricular decisions) as the CMIS Teaching Staff, with the support of the Leadership Team adopted new standards in ELA, mathematics, and science, as well as finalizing the adoption of all other content standards (i.e. physical education, social studies, health, fine arts, world languages). 

The adoption of these standards, especially in ELA, mathematics, science, and 9-12 history  reflect significant shifts in conceptual understanding of the content, have major implications on instruction and assessment, and illuminate real increases in depth of knowledge and rigor. All curricular decisions would have to be made based upon these new realities. Furthermore, CMIS has begun to view our adopted standards as the bedrock of all unit creation and planning. Standards also continue to help guide appropriate instruction and ensure rigor. 

Because of these conceptual shifts, the 2014-2015 saw a strong focus on professional development topics relating to the difference between curriculum and standards, and the shared responsibility of the CMIS Literacy Standards. Teachers, community members, and administrators were involved in multiple discussions and the CMIS Leadership Team believed this was an appropriate starting point for teachers and community members to begin understanding the standards and curriculum.
\end{findings}

\begin{evidence}
\item PTG Curriculum, Professional Development Folder \url{https://drive.google.com/a/cmis.ac.th/folderview?id=0ByVFfrm0zfolWE0yenprdktGVlk&usp=sharing}
\end{evidence}

\begin{findings}
Because of the adoption of the CMIS Literacy Standards, a great deal of training during the 2014-2015 school year focused on the ELA standards. Currently, all curriculum decisions, including resource/curriculum purchasing, must adhere to our standards with fidelity. All reading and writing curriculum decisions are based on ensuring appropriate complexity, balancing informational text, reading in the disciplines, and academic vocabulary (for reading). Writing from sources and appropriate text types are required for writing curriculum decisions. 
\end{findings}

\begin{evidence}
\item CMIS Understanding by Design Resources \url{https://drive.google.com/a/cmis.ac.th/folderview?id=0ByVFfrm0zfolfmUyZV9DbGoxZVhpVHpGdG9MeEt6MHZJaEtoT3VzTjM0bkk5NFQ5MVJldUU&usp=sharing}
\item CMIS Understanding By Design Master List
\end{evidence}

\begin{findings}

Since curricular decisions should be made using clear standards and student outcomes as a guide, a new Curriculum Review Cycle/Resource Adoption strategic plan had to be developed and implemented. The most important elements of the Curriculum Review process are the specific evaluations tools, rubrics, and vetting instruments that have been developed and used to narrow down vendors and products for curricular and resource materials. CMIS Leadership also provides the necessary time and space required for teacher collaboration in vetting and evaluation of curricular items. 

Science curriculum was the first content to be vetted and evaluated based upon the Curriculum Review Cycle; science curriculum was purchased during the 2015-2016 school year. As with ELA curriculum decisions, science saw significant conceptual changes with the adoption of the the Next Generation Science Standards. Because of this, CMIS Leadership, developed vetting and evaluation instruments to ensure curriculum and resource alignment with the standards, as well as rigor and student engagement. This year, Mathematics undertook the same curriculum vetting process and ELA/social studies will be reviewed next year. 

In order to ensure that the resources we purchase outside of the normal adoption cycle are of the highest quality and they meet our three general criteria of: is it aligned to the standards? Is it rigorous? and is it engaging? Resource Request and Resource Renewal forms were developed and implemented. 
\end{findings}

\begin{evidence}
\item "Teacher Dashboard." CMIS Curriculum. Chiang Mai International School, 22 July 2016. Web. 22 July 2016.
\end{evidence}

\begin{findings}
Science curriculum was the first content to be vetted and evaluated based upon the Curriculum Review Cycle; science curriculum was purchased during the 2015-2016 school year. As with ELA curriculum decisions, science saw significant conceptual changes with the adoption of the the Next Generation Science Standards. Because of this, CMIS Leadership, developed vetting and evaluation instruments to ensure curriculum and resource alignment with the standards, as well as rigor and student engagement. This year, Mathematics undertook the same curriculum vetting process and ELA/social studies will be reviewed next year. 

In order to ensure that the resources we purchase outside of the normal adoption cycle are of the highest quality and they meet our three general criteria of: is it aligned to the standards? Is it rigorous? and is it engaging? Resource Request and Resource Renewal forms were developed and implemented. 
\end{findings}

\begin{evidence}
\item "Model Content Frameworks." Model Content Frameworks. PARTNERSHIP FOR ASSESSMENT OF READINESS FOR COLLEGE AND CAREERS, 2016. Web. 20 July 2016. \url{http://www.parcconline.org/resources/educator-resources/model-content-frameworks}
\end{evidence}

\begin{findings}
The CMIS Google Drive is the primary method of organizing and archiving  of the curricular units. The Drive along with the the CMIS Teacher Dashboard, found on the school website, is a place where teachers can access these resources. The Dashboard also contains the CMIS Standards Blueprints. Teacher feedback and anecdotal evidence suggested that greater alignment of the UbD units to the standards would be possible if the teachers had a general blueprint or pacing guide to help them visualize and map the whole year of standards. Department and grade level teams began the blueprint work at the beginning of the 2015-2016 year. All core subjects, with the exception of K-5 ELA and mathematics, have been completed. 
\end{findings}

\begin{evidence}
\item Resource Request Form \url{https://docs.google.com/a/cmis.ac.th/forms/d/1hBKsxgHpPSOT0MtUTsZBoVX2t3ztxeGVY4R4BVNeVL8/edit?usp=sharing}
\item Resource Renewal Form \url{https://docs.google.com/a/cmis.ac.th/forms/d/1i6EnhitZ2yJZh-1OjL8jsJ2pFa7oRJCP0VcA50b_D6g/edit?usp=sharing}
\end{evidence}

\begin{findings}
ELA blueprints posed a challenge as standards are addressed all year long. In order to address this challenge, small groups were scheduled for focused professional development using the Partnership for Assessment of Readiness for College and Careers (PARCC) ELA Model Frameworks. Collaboration time was given to ELA teachers in grades 6-12 to complete this work. The CMIS Professional Development team has scheduled these same professional development opportunities for K-5 during the 2016-2017 school year. 

The combined teamwork of all stakeholders: the CMIS Leadership Team, the community, outside research, vetted educational organizations, and, most importantly,  the teachers themselves, ensure that CMIS staff at all levels stay current in educational research.
\end{findings}

\subsection{Academic Standards for Each Area}

\indicator{The school provides a comprehensive and sequential documented curriculum that is articulated within and across grade levels for the improvement of programs, learning, and teaching.}

\prompt{Evaluate to what extent there are defined academic standards for each subject area, course, and/or program (e.g., online instruction) that meet state or national/international standards.}

\begin{findings}
CMIS Leadership and Teaching Staff strongly believe in the importance of the school’s academic standards. CMIS Staff use a common understanding of academic standards: 

Academic standards define the concepts, skills, and knowledge that students should know and be able to do in each curricular area, the level at which students are expected to demonstrate this knowledge, and grade-level expectations for performance. In a standards-based educational system, schools determine the benchmarks for student work that meet these standards, provide appropriate instruction, and use multiple assessment measures to identify the level of achievement for all students. This approach assists the schools in defining the quality accomplishment of the complementary schoolwide learner outcomes and the degree to which all students are achieving them. Standards do not describe any particular teaching practice, curriculum, or assessment method 
\end{findings}

\begin{evidence}
\item Abbot, S. "Learning Standards Definition." The Glossary of Education Reform. The Great Schools Partnership, 2013. Web. 21 June 2016. \url{http://edglossary.org/learning-standards/}
\end{evidence}

\begin{findings}
CMIS provides a comprehensive and sequential documented curriculum that is articulated within and across grade levels for the improvement of programs, learning, and teaching.

CMIS teachers and students use multiple research-based standards for all grade levels and content areas. The CMIS Leadership and Professional Development Team use teacher feedback to address possible changes or modifications to the standards. 

Adopted CMIS standards include:
Common Core State Standards ELA*
Common Core State Standards Mathematics*
Next Generation Science Standards (NGSS)*
AERO Standards (aligned to NCSS framework) grades K-8*
National Standards for History (from UCLA) grades 9-12
C3 Framework (piloted 2016-2017)*
AP Course Topics and Indicators
National Coalition for Core Arts Standards  (NCCAS)
Computer Science Teachers Association Framework (CSTA)
National Educational Technology Standards for Students (NETS)
World Readiness Standards for Learning Languages
Physical Education Model Content Standards for California Public Schools 
National Health Education Standards (Center for Disease Control)
Standards for the 21st Century Learner (American Association of School Librarians)
Wisconsin’s Model Academic Standards for Business  

* standards are internationally benchmarked 

Research indicates CCSS (both ELA and Mathematics), NGSS, and C3 are standards that are considered rigorous, world class, and allow for deeper engagement around fewer concepts/topics.
\end{findings}

\begin{evidence}
\item CCSS-FAQ, Setting Criteria, Myth vs. Fact,Research Supporting Key Elements of the Standards: Appendix A. \url{https://drive.google.com/a/cmis.ac.th/file/d/0ByVFfrm0zfolcjRFZWtWWUhvR0U/view?usp=sharing}
\item NGSS-Standards Background:  Research and Reports \url{http://www.nextgenscience.org/standards-background-research-and-reports}
\item C3-Appendix C: Scholarly Rationale for the C3 Framework (p.74) \url{https://drive.google.com/a/cmis.ac.th/file/d/0ByVFfrm0zfolVGRyN0VKZzVlLVE/view?usp=sharing}
\item NCCAS-A Conceptual Framework for Arts Learning (p. 23) \url{http://www.nationalartsstandards.org/sites/default/files/NCCAS\%20\%20Conceptual\%20Framework_0.pdf}
\end{evidence}

\begin{findings}
In order to address the prior visiting team observation that some teachers did not feel that they understood the standards, CMIS has provided professional development for the CCSS ELA  and NGSS. Starting from the critical conceptual shift found in the CCSS that literacy is a “shared responsibility” of ALL staff, K-12, all contents, the CMIS Leadership Team devoted a majority of 2014-2015 to unpacking and analyzing the newly adopted ELA standards with the whole staff.  Since ALL teachers have literacy standards, the professional development ensured that teachers not only focused on their own grade-level standards (i.e “...within the grade level”), but also worked with the anchor standards that addressed vertical progression of outcomes (i.e. “...across grade levels”). 

UbD plans, mid/end of year assessment, peer reflection, and administrative observational walkthroughs are all methods to ensure instructional fidelity to the standards. 

As mentioned earlier, CMIS Leadership Team believe that all curricular decisions should be based on three basic questions: is it aligned to the standards, is it rigorous, and is it engaging? During the 2015-2016 school, the CMIS Middle and High School Science department, representative teachers from K-5, and CMIS Leadership team vetted and evaluated science materials using CMIS developed rubrics and instruments based upon EQuIP (Educators Evaluating the Quality of Instructional Products) Rubric. The outcome was the purchase of FOSS Kit science materials that met and exceeded our criteria for grades K-5. FOSS Kits for science have been purchased and CMIS will continue to provide time and assistance to help teachers utilize and implement this curriculum resource. 

This year, mathematics curriculum was reviewed and and resources were vetted. Through the use of Instructional Materials Evaluation Tool (IMET) developed by Achieve the Core, the K-5 group adopted two resources that aligned strongly with the standards and the instructional shifts in mathematics (Focus, Coherence, and Rigor). The first resource from Great Minds Inc. is entitled Eureka Math. Eureka Math not only aligns with the standards, it provides appropriate coherence, focus, and rigor. The second resource vetted and purchased was My Math published by McGraw Hill. Again, the adoption committee  found good alignment to the standards and the instructional shifts. Both programs provide a balance of rigor- application and conceptual knowledge in Eureka and procedural skills in My Math. 

Curricular decisions are also made as a result of our professional development.  After CMIS workshops on the CCSS conceptual shifts and with discussions with teachers, it was determined that the school lacked appropriately complex informational text in grades 9-12. Because of this discovery, plans and funds were set aside to purchase informational text. 

AP courses have appropriately aligned (in rigor and complexity) textbooks and resources.  

Course and grade level blueprints have also been developed to use as a guide for vertical and horizontal integration discussion and collaboration to determine if modification is necessary. Areas that have been discussed are standards overlap, skill/content gaps
\end{findings}

\subsection{Embedded Global Perspectives}
\indicator{The school leadership and certificated staff ensure that global education concepts, perspectives, and issues are embedded within the curricular areas.}

\prompt{Examine the curricular documentation and observe the delivered curriculum to determine the extent to which there is integration of global concepts, perspectives, and issues.}

\begin{findings}
The CMIS Leadership and Teaching Staff ensure that global education concepts, perspectives, and issues are embedded within the curricular areas.

CMIS Leadership and Teaching Staff wanted to describe evidence that goes beyond the traditional list of events that international schools normally cite for embedding global perspectives. Though CMIS Leadership, teaching staff, and community still organize and implement an International Day which rotates with Thai Day (i.e. every other year); other major events events such as Model United Nations and National History Day are also scheduled. Each of these events showcase and celebrate our unique place on the globe and our diverse community. There is also evidence of more routine and thoughtful examples of how CMIS promotes and embeds global perspectives. 

Most importantly, and generally overlooked, are our adopted standards. Of the core courses (ELA, Mathematics, Social Studies, and Science), all four set of standards are internationally benchmarked. This benchmarking  ensures CMIS students are exposed to global perspectives. As Achieve (2014) states about our adopted ELA and Mathematics standards, 
“As part of the Common Core State Standards Initiative, Achieve helped collect and analyze standards from a number of countries. These studies helped inform the choices made by the writers of the common standards.”
For Next Generation Science Standards, Achieve stated: 
“The overall goal of Achieve’s study on international standards is to inform the development of the NRC framework and next-generation science standards.”
And finally, College, Careers, and Civics Framework (C3) states: 
“...standards (including C3) suggests that all standards should be rigorous, world class, and internationally benchmarked, while also allowing for deeper engagement around fewer concepts/topics.” 
CMIS Leadership feel confident that as we remain focused on aligning appropriate standards to our assessments, lessons, and instruction, we can be assured that we are providing standards with a global perspective 
As illustrated in other sections, standards play a central role in planning, guiding, and developing curriculum. 
The counseling and student support departments provide multiple, targeted lessons on global perspectives. Units such as…
The diversity of our Leadership Team and Teaching Staff ensure multiple perspectives, concepts, and issues are are embedded in CMIS lessons and discussions. Diversity of not only nationality, but age, gender, and experiences makes CMIS a dynamic and engaging global workplace. 

The library provides books on multiple global issues and from a wide variety of perspectives and diverse nationalities.
\end{findings}

\begin{evidence}
\item "International benchmarking is important from a national perspective to ensure our long-term economic competitiveness." Achieve. 2012. Web. 06 May 2016. \url{http://www.achieve.org/international-benchmarking}
\item "International Science Benchmarking Report Taking the Lead in Science Education: Forging Next-Generation Science Standards Executive Summary."International Science Benchmarking Report. Achieve, 2010. Web. 20 May 2016. \url{http://www.achieve.org/files/InternationalScienceBenchmarkingReportExecutiveSummary.pdf}
\item USA. Kentucky Department of Education. Office of Next Generation Learners.704 KAR 3:303, Kentucky Core Academic Standards (First Reading). Web \url{https://drive.google.com/a/cmis.ac.th/file/d/0ByVFfrm0zfolNWh1X2w5NV9uNlU/view?usp=sharing}
\end{evidence}

\subsection{Congruence}
\indicator{There is congruence between the actual concepts and skills taught, the academic standards, and the schoolwide learner outcomes.}

\prompt{Evaluate if there is congruence between the actual concepts and skills taught, the academic standards, and the schoolwide learner outcomes.}
\begin{findings}
CMIS Leadership and teaching staff ensure congruence between the actual concepts and skills taught, the academic standards, and the schoolwide learner outcomes through a variety of methods. 

Much of the professional development and teacher work for the past two years has focused on unpacking the literacy standards as all teachers are responsible student success in reading, writing, and speaking/listening. As we focused on understanding the standards, certain structures were created to ensure fidelity to concepts/skills, standards, and SLOs. 

First, was the continuation of the Understanding by Design unit development initiative begun in 2013-2014. The traditional UbD template ensures that teacher/designers address in Stage 1: established goals, skills and knowledge, essential questions, and understandings. Stage 2: assessment, and Stage 3: learning plan. Though CMIS teachers used this general UbD template to ensure alignment with these essential, researched-based elements of UbD, a handful of modifications were added to the template to ensure congruence between our skills/knowledge, our adopted standards, and the SLOs; see Table 1 below. 

\begin{tbl}{\textwidth}{XXX}
 UbD Unit Section & Modifications to Traditional Ubd Section to Ensure Congruency& Rationale \\
\toprule[3px]
Established Goals &
Separate out Primary and Secondary Standards &
CCSS/NGSS/C3 were created to be integrated. K-12 focus on creating rich learning experiences that address multiple standards \\
\midrule
Assessment/Learning Plan &
Formative Assessment Section & 
Formative assessment was an integral part of professional development. Section created to provide place for teachers to practice. \\
\midrule
Learning Plan &
Learning Activity/ 
Experience Alignment &
Teachers were asked to explicitly align the standards to the learning experience. If not possible, they were asked why that activity was important.  \\
\midrule
Learning Plan &
Close Reading &
Anchor standard 1 asks students to “read closely”. Close reading is an outcome important to students in grades K-12. \\ 
\midrule
Learning Plan &
Complexity Level &
Shift 3 of CCSS ELA requires a “staircase of complexity”. Appropriate complex text for each grade level is essential to complete this shift.  \\
\midrule
Topic Section &
SLO &
Teachers and Leadership felt strongly that the most effective way to teach SLOs was implicitly, through the UbD units. \\
\bottomrule
\end{tbl}

Secondly, CMIS Leadership and Teaching Staff have implemented the Datawise Process. Using this process, CMIS has created school data teams of teachers and administrators who make use of performance data and other information to target educational questions to pursue, identify major gaps in student understanding, identify target areas called learner-centered problems (LCP), reframe learner-centered problems as problems of practice (POP), target solutions to problems of practice, and write action plans pinpointing how broadly solutions will be implemented, how they can be implemented and adapted across grade levels and content areas, and on what timelines they will be carried out. 
\begin{evidence}
\item McTighe, Jay, and Elliot Seif. "A Summary of Underlying Theory and Research Base for Understanding by Design by Jay McTighe and Elliott Seif." A Summary of Underlying Theory and Research Base for Understanding by Design by Jay McTighe and Elliott Seif. Jay McTighe Educational Associates, 2002. Web. 4 Apr. 2016. \url{http://jaymctighe.com/wordpress/wp-content/uploads/2011/04/UbD-Research-Base.pdf}
\item Nel has photos
\item Houdett, Kathryn Parker, Elizabeth A. City, and Richard J. Murnane. Data Wise. Cambridge: Harvard Education, 2013. Print. \url{http://isites.harvard.edu/icb/icb.do?keyword=datawise&pageid=icb.page555331}
\end{evidence}

CMIS Leadership and Teaching Staff used 2014-2015 ISA data to develop a CMIS specific schoolwide LCP, created a schoolwide POP, each department created a strategy to address the POP, and assessed throughout the 2015-2016. Based upon teacher feedback and achievement data, the 2016-2017 year Datawise program was modified to allow each department to determine their LCP and POP based upon department specific data. The departments’ Datawise plan is currently being implemented. By using the Datawise process, the CMIS Leadership the Teaching Staff have ensured academic outcomes (i.e. standardized test data) are used to make instructional decisions that were based upon students’ skills/ concepts, and standards. 

Looking at student work

Review the SLOs during teacher orientation (2016-2017) and plan implementation for review with students 
\end{findings}

\begin{evidence}
\item McTighe, Jay, and Elliot Seif. "A Summary of Underlying Theory and Research Base for Understanding by Design by Jay McTighe and Elliott Seif." A Summary of Underlying Theory and Research Base for Understanding by Design by Jay McTighe and Elliott Seif. Jay McTighe Educational Associates, 2002. Web. 4 Apr. 2016. \url{http://jaymctighe.com/wordpress/wp-content/uploads/2011/04/UbD-Research-Base.pdf}
\item Nel has photos
\item Houdett, Kathryn Parker, Elizabeth A. City, and Richard J. Murnane. Data Wise. Cambridge: Harvard Education, 2013. Print. \url{http://isites.harvard.edu/icb/icb.do?keyword=datawise&pageid=icb.page555331}
\end{evidence}
\section{Support for Student and Personal Academic Growth}
\section{Resource Management and Development}
\chapter{Schoolwide Action Plan}
\chapter{Appendix}
%\listoffigures
\end{document}
