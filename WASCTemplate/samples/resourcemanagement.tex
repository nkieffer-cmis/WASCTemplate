\section{Resource Management and Development}

\subsection{D1 Resources Criterion}

The resources available to the school are sufficient to sustain the school program and are effectively used to carry out the school’s purpose and student achievement of the schoolwide learner outcomes, i.e., global competencies.

\subsubsection{Allocation Decisions}

\indicator{There is a relationship between the decisions about resource allocations, the school’s vision, mission and student achievement of the schoolwide learner outcomes and the academic standards. The school leadership and staff are involved in the resource allocation decisions.}

\prompt{To what extent are resources allocated to meet the school’s vision, mission, and student achievement of the critical learner needs, the schoolwide learner outcomes and the academic standards? Additionally, comment on the extent to which leadership and staff are involved in the resource allocation decisions. What impact has the process for the allocation of resources made on student learning?}

\begin{findings}
Chiang Mai International School (\href{http://cmis.ac.th/}{CMIS}) continues to develop procedures and processes to ensure that the allocation of resources is in line with the \href{http://cmis.ac.th/about/vision}{school vision, mission statement}, student learning objectives (SLOs), curricular standards, and core values. 

The School Manager is responsible for the financial affairs of the school, human resources, purchasing, maintenance, gardening, and security with the help of the Assistant Manager. The Manager also oversees the functions of the office and management of the school budget along with the School Executive Team (SET). Through a standardized budgeting process, The School Executive Team (SET) prioritizes the most important education areas based on school-wide initiatives and divisional needs. The School Management Team (SMT), Teacher Leadership Team, and Directors are involved in the resource allocation decisions. 

CMIS Leadership has developed a \href{https://docs.google.com/document/d/1hh1nLUlJgg1hd7s6aG3u3We0L6o7Wg_ECdjc2f6DcT8/edit}{Curriculum Review} process in which specific evaluations tools, rubrics, and vetting instruments are used used to narrow down vendors and products for curricular and resource materials. CMIS Leadership also provides the necessary time and space required for teacher collaboration in vetting and evaluation of curricular items. CMIS Leadership will only allocate money for resources that are aligned to the standards, rigorous, and engaging and directly related to student achievement.  

The primary areas of budgetary allocation and expense (in the General Fund) are Salary and Benefit (70\%), Instructional Expenses (7\%), Education Support (6\%), Utilities and Maintenance (5\%), Furniture and Equipment (2\%), and Student Welfare (1\%). 

CMIS is aligned with The Church of Christ in Thailand (CCT) to establish the budget cycle, every October for the preliminary budget for the next school year and for the Final budget for the current school year.

Tuition and fees account for 99\% of annual income. For the past three years, tuition has increased at approximately 6-7\% annually for Standard and Discount Groups and 5\% for the Missionary Group. CMIS is a not-for-profit school; our tuition and fees are in the average range compared to other international schools in Chiang Mai for the Standard category and in the low range for the Missionary category.

The annual community survey, which is sent to parents, teachers, and students, is closely aligned with the school mission. This survey is given to all internal stakeholders (students, staff, and parents). The result of the survey showed that 42\% of ES teachers and 64\% of MS and HS teachers have adequate IT support and resources.  In an effort to increase this area, the IT department has added Chromeboxes in the upper elementary classrooms and is sponsoring a Google Certified Educator Level 1 certification support for all staff. In addition, the IT working with the ES division to identify other ideas of how to further support them. At the time of the survey Middle School and High School students in grades 7, 8, and 9 received Chromebooks (as the second year of a three year implementation of 6-12). From the 2016-17 academic year forward, grades 6-12 receive school-provided Chromebooks. The 6-12 Chromebook program brings technology opportunities into all secondary classrooms and reduces the load on the elementary computer lab (previously used for middle school computer classes), allowing more access time to elementary students. Additionally, the IT department has moved to a ticketing system for support requests. These changes and additions should bring a positive impact to the need for IT support and resources.

\minor{So what...}

CMIS Leadership continues to involve multiple stakeholders in the resource allocation process. Steps have been taken to ensure resources are of high quality and are aligned to our standards. Though CMIS will continue to focus on resource allocation transparency, data indicates that more time and resources could be spent on improving the purchasing process and communication between teachers, leadership, and the purchasing department.  
\end{findings}

%% \subsubsection{Practices}

%% \indicator{The school develops an annual budget, has an annual audit, and at all times conducts quality business and accounting practices, including protections against mishandling of institutional funds.}

%% \prompt{Evaluate the school's processes for developing an annual budget, conducting an annual audit, and at all times conducting quality business and accounting practices, including protections against mishandling of institutional funds.}

%% \begin{findings}
%% CMIS has a clearly articulated, standardized budget process, ranging from August to July to comply with the Church of Christ in Thailand’s policy. 

%% \minor{Annual Budget}

%% The annual budget development is the responsibility of the School Manager and approved by the SET and the Board. During the budgeting process, the Manager seeks input from SMT, HODs, and Directors. The Preliminary Budget for the next school year has to be approved by the SET and sent to the Board for approval and then to the Foundation of the Church of Christ in Thailand (CCT) for approval in October. The Final Budget of the current school year has to be approved by the SET, then sent to the Board, and finally to the CCT in the same month.

%% Within this budget process, the Board has a long-term projection on enrollment that determines the staffing of full-time equivalent (FTE) and part-time teacher and staff. The revised organization structure for the next school year is presented to the Board in February - March. For foreign staff, the Superintendent normally sends a letter of intent in December and asks them to reply by January the next year which allows foreign hire employment contracts to be issued by late January in order to finalize the school’s recruiting needs. The job posting will be started from February until the positions are filled.

%% Once the budget is approved by the CCT, the SET is responsible for its implementation and the Manager monitors day-to-day budget expenditures and other financial matters. Monthly Financial Reports are sent to the Board for approval. The Board is not involved in the day-to-day financial or operational affairs of the school.

%% The Finance Advisory Committee is approved by the Board every three years. The committee meets and makes a recommendation to the Board to approve tuition and fees for the three-year scheme. After the Board approves, the Tuition scheme will be announced to parents. The Finance Advisory Committee is comprised of the SET, one parent representative from Missionary group, one parent representative from a non-Missionary group, Head of Finance Department, and one Board member.

%% The School Executive Team search data from member schools of East Asia Regional Council of Schools (\href{https://www.earcos.org/}{EARCOS}), International School Association of Thailand (\href{https://www.isat.or.th/index.php}{ISAT}), and Chiang Mai Circle of International Schools (CMCIS, ISCN) for salary and benefits data from benchmark schools to determine that CMIS’s salary and benefits (S\&B) are fair and equitable. 

%% For salary and benefits, the SET discusses and finalizes the next two-year plan with Teacher and Administrative  Communication Team (\href{https://docs.google.com/document/d/14nhwcw8xo3i-23Q-WUxo6KJ_c8yFKu-jTdCctt4MFcs/edit}{TACT}) for foreign staff and with the Head of Departments for Thai staff. Then the SET recommends it to the Board to approve before sending an announcement to all staff in December.

%% \minor{External and Internal Audits}

%% Annually, CMIS conducts external audits by an external firm approved by the CCT in October and by the CCT Internal Auditor in November. After both groups approve the audit, the SET will recommend to approve the allocation of Net Income Over Expenses to the Board. Net Income Over Expenses is then reinvested into the CMIS Specific Funds (i.e. retirement, human resources, and emergency)and Plant Fund (i.e. building, furniture, and technology). The external and internal audits also prevent unintentional misuse of CMIS funds. 

%% \minor{Quality Business and Accounting Practices}

%% The Board performs the following duties to safeguard the financial stability of CMIS:

%% \begin{itemize}
%% \item approves monthly and annual financial reports 
%% \item approves clear policies recommended by the SET and changes in any policies
%% \item approves school-wide investment
%% \item monitors processes and procedures for business operations to comply with the CCT policy.
%% \end{itemize}

%% \minor{So what...}

%% CMIS has several internal and external structures in place to ensure the development of an annual budget, an annual audit, and at all times conducts quality business and accounting practices, including protections against mishandling of institutional funds.

%% CMIS Leadership should continue to maintain and monitor CMIS auditing, budgetary, and accounting practices. 
%% \end{findings}
