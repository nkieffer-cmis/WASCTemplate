\usepackage[most]{tcolorbox}
\usepackage{marginnote}
\usepackage{booktabs}
\usepackage{titlesec}
\newcommand{\debug}[1]{%
        \newif\ifdebugon
        \debugonfalse
        \ifdebugon
        \marginnote{%
                \begin{debugBox}
                        \textbf{\showlineno}
                        \\
                        #1
                \end{debugBox}
        }%
        \fi
}

\newtcolorbox{evidenceBox}{%
  colback=black!10,
  %colbacktitle=violet,
  boxrule=0pt,
  top=1em,
  breakable,
  arc=0pt,
  outer arc=0pt,
 % title=Evidence
}

\newtcolorbox{findingsBox}{%
  colback=white,%black!5,
  colbacktitle=red,
  boxrule=1pt,
 % boxsep=1pt,
 % top=1em,%-0.5\baselineskip,
  breakable,
 % arc=4pt,
 % outer arc=4pt,
  title=Findings
}

\newtcolorbox{debugBox}{%
        colback=yellow,
        title=debug
}
        


\newenvironment{evidence}
{
%\debug
\begin{evidenceBox}
\vspace{-\topsep}
\noindent\textbf{Evidence}

\begin{itemize}[leftmargin=*]
  \setlength{\parskip}{0pt}
  \setlength{\itemsep}{1pt}
}
{
\end{itemize}
\end{evidenceBox}
%\vspace{4em}
}

\newcommand{\placeholder}[1]{
\begin{tcolorbox}[title=CONTENT PLACEHOLDER]
#1
\end{tcolorbox}
}


\newenvironment{tbl}[4]{ %1 = width, 2 = col format, 3 = caption, 4 = label
%\rowcolors{2}{lightgray}{white}
\newcommand{\tblcap}{\captionof{table}{\textbf{#3}}\label{#4}}
\begingroup
\tabularx{#1}{#2}
%\rowcolor{darkgray}
}
{
\endtabularx
\tblcap
\endgroup
}

\newcommand{\blockquote}[1]{
``#1''
}

\let\oldtoc\tableofcontents
\renewcommand\tableofcontents{
\setlength{\parskip}{0em}
\oldtoc
\setlength{\parskip}{1em}
}

\let\oldlof\listoffigures
\renewcommand\listoffigures{
\setlength{\parskip}{0em}
\oldlof
\setlength{\parskip}{1em}
}

\let\oldlot\listoftables
\renewcommand\listoftables{
\setlength{\parskip}{0em}
\oldlot
\setlength{\parskip}{1em}
}

\definecolor{CmisPurple}{HTML}{885190}


\newcommand{\minor}[1]{
\noindent\textbf{\sffamily\color{black!25!CmisPurple}#1}
}

\newenvironment{findings}
{
\minor{ Findings}

}
{
%\end{findingsBox}
}


\newcommand{\indicator}[1]{ 
\noindent{\textbf{\sffamily \color{black!25!CmisPurple}Indicator:}} #1
}

\newcommand{\prompt}[1]{
 
\noindent{\textbf{\sffamily \color{black!25!CmisPurple}Prompt:}} \textit{#1}
}


\newtcolorbox{sectionbox}{
coltext=black,
colback=CmisPurple!50, 
colframe=CmisPurple,
arc=0mm,
%size=small
}

\newcommand\tcbsection[1]{%
%\begin{sectionbox}%
\section{#1}%
%\end{sectionbox}
}

\newtcolorbox{chapterbox}{
coltext=CmisPurple,
boxsep=0pt,
%colback=CmisPurple!50, 
colframe=CmisPurple,
arc=0mm,
%size=small
}

%\usepackage[explicit]{titlesec}
%\titleformat{\chapter}[display]
%\titleformat{\chapter}
%    {\color{white}\normalfont\huge\bfseries}{\chaptertitlename\ \thechapter}{20pt}{\Huge}
\titleformat{\chapter}
%{\newgeometry{left=0in, right=0pt,top=0pt,bottom=0pt}
{\thispagestyle{empty}\Huge\bfseries\raggedright}
{}
{0em}
{\colorbox{CmisPurple}{\parbox{\dimexpr\textwidth-2\fboxsep\relax}
{\hyphenpenalty=10000\raggedright\textcolor{white}{\sffamily#1}}}%
\begin{tikzpicture}[remember picture, overlay]
\filldraw[fill=CmisPurple,draw=CmisPurple] (1,0) rectangle (3, -5);
\end{tikzpicture}}

\titleformat{\section}
{\sffamily\bfseries\huge\color{CmisPurple}}
{}
{0em}
{\includegraphics[height=0.75em]{triangle.png} #1}

\titleformat{\subsection}
{\sffamily\bfseries\Large\color{CmisPurple}}
{}
{0pt}
{#1}[\color{CmisPurple}\titlerule]

\titleformat{\subsubsection}
{\sffamily\bfseries\large\color{CmisPurple}}
{}
{0pt}
{#1}

%\titlespacing*{\chapter}{0pt}{5.5in}{0pt}
\newcommand\tcbchapter[1]{%
%\begin{chapterbox}
\chapter{#1}
%\end{chapterbox}
}


\usepackage{eso-pic}

 \newcommand\BackgroundPic[1]{%
   \put(0,0){%
     \parbox[b]{\paperwidth}{%
       \vfill
       \centering
       \includegraphics[width=\textwidth,%\paperwidth,%height=\paperheight,%
         keepaspectratio]{#1}%
       \vfill
 }}}

\newcommand\CoverPic[1]{%
   \put(0,-150){%
     \parbox[b][\paperheight]{\paperwidth}{%
       \vfill
       \centering
       \includegraphics[width=\paperwidth,%height=\paperheight,%
         keepaspectratio]{#1}%
       \vfill
 }}}
