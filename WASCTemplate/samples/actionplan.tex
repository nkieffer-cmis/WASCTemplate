\section{FUTURE FOCUS AREA 1: CMIS Data Sandbox/Dropbox }

CMIS needs to develop, maintain, and use a central data collection, visualization, and archival system in order for key stakeholders to easily access and analyze appropriate data to make informed instructional, curriculum, assessment, and resource decisions. 

\minor{Rationale}

Chiang Mai International School has the classic DRIP Syndrome. We, like many other international schools, are data rich but information poor (Dufour, Dufour, Eaker, \& Many, 184). CMIS has several processes and procedures that collect information but do not have the systems in place to easily translate the data into useable information. Our issue is not acquiring data, it is learning to sift through it; collecting information that is relevant to achieving our mission. Being data savvy is no longer reserved for a few select administrators, it is necessary for everyone who wants to better serve their learners. CMIS needs an easy to use, easy to access data system as it affects all programs school wide. \ref{table:1}

\minor{SLO Addressed}

Pursue personal growth as adaptive, independent learners, utilize resources and technology to effectively support learning and work 
\begin{landscape}
\begin{table}[h]
\centering
\caption{CMIS Data Sandbox Timeline}
\label{table:1}
\begin{tabu} to 8in {|X|X|X[2]|X[2]|X[2]|X|}
\hline
TASKS &
TIMELINE &
PERSON(S) RESPONSIBLE &
STRATEGIES and RESOURCES &
ASSESSMENT &
METHODS OF REPORTING  \\
\hline

Create a data inventory &
August 2017 &
\parbox[t]{4cm}{
SET (School Executive Team) \\ 
IT Department \\
Teacher Leadership Team (TLT) \\ 
Division principals}  &
\parbox[t]{4cm}{
Internal interviews \\
Determine categories of data \\
Determine essential information to be inventoried per data set} &
Completed inventory &
Report to SET \\
\hline

Research data systems &
September 2017 &
IT Department &
Look at like school systems and procedures &
Report on research based data systems &
Report to SET \\
\hline

Develop Access Plan &
&
\parbox[t]{4cm}{
SET \\
Teacher Leadership Team \\
IT Department \\
Division principals} &
Data Inventory  &
Report on Access Plan &
Report to SET \\
\hline
\savetabu{timeline}
\end{tabu}
\end{table}
\end{landscape}

\begin{landscape}
\begin{table}[h]
\caption{CMIS Data Sandbox: SLO Addressed}
\label{table:2}
\centering
\begin{tabu} to \linewidth {|X[1]|X[5]|}
\hline
Category &
\parbox[t]{6in}{ 
\minor{Indicator}\\ 
Comment on the degree to which this criterion impacts the school’s ability to address one or more of the identified critical learner needs.}\\
\hline
\minor{Organization for Student Learning} &
\parbox[t]{6in}{
\minor{Degree: HIGH}\\
This category contains a great deal of data that should be accessed by the appropriate stakeholders, such as: community, student, and teacher survey, governance/board perception, teacher certification and qualification data. } \\
\hline
\minor{Curriculum, Instruction, and Assessment} &
\parbox[t]{6in}{
\minor{Degree: HIGH}\\
This category contains data that is essential to effective instruction, assessment, and curriculum development. Test data, from assessments such as: ISA, MAP, AP, DRA, and WIDA are important to student placement and progress reporting. Curriculum data from vetting instruments, UbD units, looking at student work reflections would be archived in the Data Sandbox. Teach for Success, Focus Questions from Instructional Rounds, and Datawise data needs to be accessible as well. } \\
\hline
\minor{Support for Student Personal and Academic Growth } &
\parbox[t]{6in}{
\minor{Degree: HIGH} \\
Data from this category includes student perceptions, community and student involvement in after school activities, club, and student support activities, including community service hours and options are all data sets that could be accessed from the Sandbox.} \\
\hline
\minor{Resource Management and Development } &
\parbox[t]{6in}{
\minor{Degree: HIGH} \\
Data from curriculum reviews and vetting instruments, budgetary and fiscal topics, campus development, and professional development could be stored and accessed appropriately from the Sandbox.}\\ 
\hline
\multicolumn{2}{|l|}{
\parbox[t]{\textwidth}{
\minor{So what...}
CMIS has recently strengthened the student admissions process and is currently experiencing maximum enrollment. In order to maintain capacity and remain competitive, CMIS needs to find additional ways to market CMIS in the community.
}}\\
\hline
\savetabu{summary}
\end{tabu}
\end{table}
\end{landscape}

\section{FUTURE FOCUS AREA 2: Marketing}

CMIS should continue to find marketing strategies that are research based and include current enrollment trends to elicit the recruitment and retention of highly committed students. This would enable CMIS to remain competitive in establishing and promoting an international balance of students receiving academic excellence in a caring environment where students are challenged and successful. 

\minor{Rationale}

CMIS is the educational leader in northern Thailand. CMIS should continue find additional ways to promote itself and share how CMIS students excel in many areas including the arts, athletics, academics, community service, and leadership; as well as the expertise and professionalism of the teaching staff. Developing a plan that reviews future trends in international schooling, community demographics, and local competition to create marketing strategies for attracting  students committed to the SLO’s will ensure the future success of CMIS. 

\minor{SLO Addressed}

Embody a work ethic that values learning and academic integrity, pursue personal growth as adaptive, independent learners, serve as responsible, proactive members of the global community

\begin{landscape}
\begin{table}[h]
\centering
\caption{CMIS Marketing Timeline}
\label{table:3}
\begin{tabu} {\usetabu{timeline}}
\hline
TASKS &
TIMELINE &
PERSON(S) RESPONSIBLE &
STRATEGIES and RESOURCES &
ASSESSMENT &
METHODS OF REPORTING  \\
\hline
Further develop marketing plan &
July 2018 &
SMT and PR committee &
\parbox[t]{4cm}{
Research-based knowledge \\
Marketing Workshop\\
Alumni} &
\parbox[t]{4cm}{
Complete Marketing Plan \\
Implement Marketing Plan} &
Report to the Board \\
\hline
Analyze demographic data &
December 2017 &
SMT and PR committee &
\parbox[t]{4cm}{
Community survey data \\
Survey data \\
Student Community Profile data \\
Powerschool} &
\parbox[t]{4cm}{Complete analysis of demographics and trends\\
Monitor enrollment trends at CMIS and other Chiang Mai international schools } &
Report to the Board \\
\hline
Profile and analyze Chiang Mai expatriate community &
December 2017 &
PR committee &
\parbox[t]{4cm}{
Survey\\
Statistic data\\
Focused group} &
Complete profile & 
Report to SET \\
\hline
Identify CMIS philosophy
and share with community &
July 2017 & 
SMT &
\parbox[t]{4cm}{
Powerschool \\
Student Community Profile data \\
Community survey \\
Advertising }&
\parbox[t]{4cm}{
Complete characteristics checklist  \\
Meet enrollment targets} &
- \\
\hline
Analyze admissions data &
ongoing &
Admissions &
\parbox[t]{4cm}{
Powerschool \\
ISA data \\
MAP data \\
AP data } &
Complete admissions report &
Report to SET \\
\hline
Continue to fine tune ELL admissions criteria &
ongoing &
\parbox[t]{4cm}{
SET  \\
Student Success Team} &
\parbox[t]{4cm}{
WIDA test \\
Admissions criteria \\
ELL Coordinator position} &
Establish guidelines &
Report to SET and the Board \\ 
\hline
\end{tabu}
\end{table}
\end{landscape}

\begin{landscape}
\begin{table}[h]
\centering
\caption{CMIS Marketing Summary}
\label{table:4}
\begin{tabu} {\usetabu{summary}}
\hline
Category &
\parbox[t]{6in}{ 
\minor{Indicator}\\ 
Comment on the degree to which this criterion impacts the school’s ability to address one or more of the identified critical learner needs.}\\
\hline
Organization for Student Learning &
\parbox[t]{6in}{ 
\minor{Degree: HIGH}\\ 
CMIS mission is to develop learners who can pursue personal and academic goals, based on educational excellence, and to equip international students for lives of learning and positive contributions locally and globally. It cannot do this if it does not attract and retain motivated students and families who share the same belief. 
}\\
\hline
Curriculum, Instruction, and Assessment &
\parbox[t]{6in}{ 
\minor{Degree: HIGH}\\ 
In order to maintain CMIS as the international school leader in northern Thailand, the curriculum, instruction, and assessment needs to continue to be of the highest quality. In order to continue to attract highly committed students only resources that are engaging, aligned to the standards, and rigors will be considered for adoption at CMIS. Instruction needs to continue to be engaging, student-centered, and research based. Finally, assessment must continue to focus on student growth and college and career readiness. }\\
\hline
Support for Student Personal and Academic Growth  &
\parbox[t]{6in}{ 
\minor{Degree: HIGH}\\
Student Support and Student Life are closely tied to athletic and community service programs that CMIS would like to celebrate as an avenue to promote student achievement.  }\\
\hline
Resource Management and Development &
\parbox[t]{6in}{ 
\minor{Degree: HIGH }\\
Resource management and development is closely tied to enrollment. As 99\% of the CMIS budget is correlated to tuition fees the recruitment and enrollment of highly committed students is essential.}\\
\hline
\multicolumn{2}{|l|}{
\parbox[t]{8in}{
\minor{So what...}\\
CMIS has recently strengthened the student admissions process and is currently experiencing maximum enrollment. In order to maintain capacity and remain competitive, CMIS needs to find additional ways to market CMIS in the community.
}}\\
\hline


\end{tabu}
\end{table}
\end{landscape}

\section{FUTURE FOCUS AREA 3: Staffing}

CMIS should continue to recruit, develop, and retain highly qualified teachers, administrators, and staff who are committed to our mission and vision. 

\minor{Rationale}

The CMIS Teaching Staff is one of our most important resources. The success of our programs depends on the skill and expertise of highly qualified staff. CMIS should continue to promote the benefits of working at CMIS beyond the financial compensation: caring Christian community, positive learning environment, research based professional development, leadership opportunities, substantial planning time, small class sizes, highly qualified staff, beautiful location, and reasonable cost of living. The ongoing success of CMIS requires the recruiting, developing, and retaining of highly qualified staff members who are committed to CMIS mission and educational goals.

\minor{SLO Addressed}

Understand Christian virtues and positive student character, demonstrate integrity through consistent respect for people of all faiths, develop cultural awareness and an appreciation for diversity, serve as responsible, proactive members of the global community

\begin{landscape}
\begin{table}[h]
\centering
\caption{CMIS Staffing Timeline}
\label{table:5}
\begin{tabu} {\usetabu{timeline}}
\hline
TASKS &
TIMELINE &
PERSON(S) RESPONSIBLE &
STRATEGIES and RESOURCES &
ASSESSMENT &
METHODS OF REPORTING  \\
\hline
%--------------------------------%
Identify key characteristics and attributes of effective CMIS teachers&
July 2017 &

\parbox[t]{4cm}{
Teacher Leadership Team (TLT) \\
Divisional Principals \\
Superintendent} &

\parbox[t]{4cm}{
Research based best practice dispositions of CMIS effective staff \\
Ensure understanding of CMIS professional development, contract time, instructional and assessment expectations \\
Ensure CMIS standards based instruction expectations \\
Exit interviews \\
Common interview questions \\
Focused group } &

\parbox[t]{4cm}{
Post characteristics on website \\
Complete CMIS Staff Dispositions framework \\
Implement framework } &

\parbox[t]{4cm}{
Report to SET \\
Develop rubric for hiring CMIS teaching staff} \\
\hline
%--------------------------------%



Provide regionally competitive salary and benefits package &

December 2017 &

\parbox[t]{4cm}{
SET \\
TACT } &

\parbox[t]{4cm}{
Comparison of retention rates data  \\
Salary and Benefit package data of schools in the region } &

\parbox[t]{4cm}{
EARCOS salary survey data \\
Academy of International Heads salary survey } &

Report to the Board \\
\hline

%--------------------------------%

Ensure smooth transition and ongoing support of new teachers to CMIS, Chiang Mai, and Chiang Mai International School &

Ongoing &

\parbox[t]{4cm}{
CMIS Buddies Team New Staff committee \\
SMT \\ 
HR Department } &

\parbox[t]{4cm}{
New teacher orientation schedule \\
Contact with TLT } &

New Staff Orientation survey  &

- \\
\hline
%--------------------------------%

Deliver and train new staff with consistent and research based evaluation of new staff &

Ongoing &

\parbox[t]{4cm}{
Superintendent \\  
Divisional Principals} &

\parbox[t]{4cm}{
The Essential Practices of High Quality Teaching and Learning \\
Instructional Rounds plan \\
Schedule of formal observations and evaluations} &

Schedule of formal observations and evaluations &

Report to SET \\

\hline
%--------------------------------%

Develop and deliver professional development that “catches” new staff up-to-speed with CMIS initiatives &

Ongoing &

\parbox[t]{4cm}{
Superintendent \\
Divisional Principals } &

\parbox[t]{4cm}{
New teacher orientation schedule \\
Professional Development Plan} &

\parbox[t]{4cm}{
Annual reflection \\  
Evaluation of plans and instruction by new staff }&

Report to SET \\
\hline
%--------------------------------%


\end{tabu}
\end{table}
\end{landscape}

\begin{landscape}
\begin{table}[h]
\centering
\caption{CMIS Staffing Summary}
\label{table:6}
\begin{tabu} {\usetabu{summary}}
\hline
Category &
\parbox[t]{6in}{ 
\minor{Indicator}\\ 
Comment on the degree to which this criterion impacts the school’s ability to address one or more of the identified critical learner needs.}\\
\hline
Organization for Student Learning &
\parbox[t]{6in}{ 
\minor{Degree: HIGH}\\ 
This criterion greatly impacts this critical area. CMIS believes that student learning benefits greatly from a highly qualified and trained team, and school leadership is committed to continuing to recruit highly qualified and effective staff to advance the school’s purpose and core values. 
}\\
\hline
Curriculum, Instruction, and Assessment &
\parbox[t]{6in}{ 
\minor{Degree: HIGH}\\ 
Professional development of new and existing staff and evaluation are key considerations of this critical area. As such, the degree with which the Curriculum, Instruction, and Assessment criterion impacts the ability to address this area is high. Through the continued implementation of Instructional Rounds to develop individualized professional development, to ensuring that research based evaluation standards are being met, to providing the New Teacher Orientation learning modules at the beginning of the year, this criterion greatly impacts this critical area. }\\
\hline
Support for Student Personal and Academic Growth  &
\parbox[t]{6in}{ 
\minor{Degree: HIGH}\\
Highly qualified staff are essential in order to promote student achievement and preparation for global competitiveness by fostering educational excellence and ensuring equal access. This criterion greatly impacts this critical area.   }\\
\hline
Resource Management and Development &
\parbox[t]{6in}{ 
\minor{Degree: HIGH }\\
One of CMIS most important resources in their highly qualified and dedicated staff. Effective management of this resource is necessary to effectively attract, recruit and retain educators that can implement the mission of the school.}\\
\hline
\multicolumn{2}{|l|}{
\parbox[t]{8in}{
\minor{So what...}\\
CMIS staff are highly qualified and well trained in their pursuit of best instructional practice. This level of staff excellence and leadership support needs to constantly maintained and strengthened. Increasing student achievement is a continuous effort and cannot be accomplished without collaboration and shared accountability.

}}\\
\hline


\end{tabu}
\end{table}
\end{landscape}

\section{FUTURE FOCUS AREA 4: External Resources}

CMIS should continue to find alternative opportunities of funds from external sources. 

\minor{Rationale}
Currently, CMIS is funded primarily by tuition. This model is not sustainable for the future development of CMIS. In order to have enough reserve to remain competitive, CMIS needs to look for external sources of funds to support future initiatives . 

\minor{SLO Addressed}

Utilize resources and technology to effectively support learning and work

\begin{landscape}
\begin{table}[h]
\centering
\caption{CMIS External Resources Timeline}
\label{table:7}
\begin{tabu} {\usetabu{timeline}}
\hline
TASKS &
TIMELINE &
PERSON(S) RESPONSIBLE &
STRATEGIES and RESOURCES &
ASSESSMENT &
METHODS OF REPORTING  \\
\hline
%--------------------------------%
Networking & 

Ongoing &
\parbox[t]{4cm}{
The Board \\
SET \\
Alumni Relation \\
Fundraising Committee} &
Membership of  ISAT, EARCOS, ISCN, and CMCIS
PTG &
\parbox[t]{4cm}{
Brand awareness \\
Connections
} &
Report to the Board \\
\hline
%--------------------------------%
PR and Marketing &
Ongoing &
\parbox[t]{4cm}{
PR committee \\
SMT } &
\parbox[t]{4cm}{
PR and Marketing Plan \\ 
Budget Plan \\
Survey Data } &
Brand awareness &
Report to the Board \\

\hline
%--------------------------------%
Fundraising &

Ongoing &

\parbox[t]{4cm}{
Fundraising committee \\
Alumni Relation \\
SET \\
Board Chair } &

\parbox[t]{4cm}{
The Board direction and goal \\
IT and resources \\
Alumni Relation Office }&

\parbox[t]{4cm}{
Fundraising achievement \\
Alumni networking}&

Report to the Board \\
\hline
%--------------------------------%
\end{tabu}
\end{table}
\end{landscape}

\begin{landscape}
\begin{table}[h]
\centering
\caption{CMIS External Resources Summary}
\label{table:8}
\begin{tabu} {\usetabu{summary}}
\hline
Category &
\parbox[t]{6in}{ 
\minor{Indicator}\\ 
Comment on the degree to which this criterion impacts the school’s ability to address one or more of the identified critical learner needs.}\\
\hline
Organization for Student Learning &
\parbox[t]{6in}{ 
\minor{Degree: HIGH}\\ 
School tuition currently funds 99\% of the CMIS budget. Teacher compensation is more than salary; it is a valuable total package that includes salary, benefits, and insurance. Combined, they are the single largest expenditure at most schools and at CMIS. In order to provide new campus and resource development funds outside of the tuition base will be necessary. 
}\\
\hline
Curriculum, Instruction, and Assessment &
\parbox[t]{6in}{ 
\minor{Degree: HIGH}\\ 
Curriculum, instruction, and assessments require continuous evaluation, renewal, and refinement. In order to implement a long term strategic plan for academic excellence, additional fund sources need to be found. CMIS instruction needs to remain rigorous, coherent, and focused to continue to be recognized as the forerunner of educational excellence in the North of Thailand. Investing in up to date technologies, resources, and materials will continue to be crucial elements in the future development of CMIS. Obtaining resources that are not tied directly to tuition will add more flexibly to the implementation of strategic planning. }\\
\hline
Support for Student Personal and Academic Growth  &
\parbox[t]{6in}{ 
\minor{Degree: HIGH}\\
CMIS believes that all students are unique and works hard to ensure that their diverse emotional, academic, and social needs are met. This often requires additional staffing, resources, and materials. In order to implement a long term strategic plan to support students’ personal and academic growth, additional fund sources need to be found.   }\\
\hline
Resource Management and Development &
\parbox[t]{6in}{ 
\minor{Degree: HIGH }\\
External funds could be used for future improvement of school facilities, IT and other resources. It also can be used for 
increase in future salary and benefits to be remain competitive with local international schools.}\\
\hline
\multicolumn{2}{|l|}{
\parbox[t]{8in}{
\minor{So what...}\\
In order to implement successful strategic plans and meet desired goals, CMIS needs to find additional ways to generate funding in addition to tuition. 
}}\\
\hline


\end{tabu}
\end{table}
\end{landscape}
